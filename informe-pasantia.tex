% Centro de Estadística y Matemática Aplicada
% Universidad Simón Bolívar
% Plantilla LaTeX para manuscritos (tesis y pasantías)
% pregrado y postgrado
%
% Andrés M. Sajo-Castelli
% Carlos Contreras
%
% 15 Abril 2015 -- primera versión pública
% ...
% 11 Mayo 2018 --- Se agrega bibliografía en castellano via babelbib
%
\documentclass[pregrado]{tesis-usb}

% paquetes
\usepackage[utf8]{inputenc}
\usepackage{verbatim}
\usepackage{acronym}
\usepackage{amsmath}
\usepackage{amsfonts}
\usepackage{amssymb}
% \usepackage{hyperref}

% estilo de las referencias
\usepackage[fixlanguage]{babelbib-and}\selectbiblanguage{spanish}
\usepackage{url}
\bibliographystyle{babplain-lf}

\autor{Ricardo Münch}
\autori{R. Münch}
\usbid{11-10684}
\titulo{Desarrollo de la versión 2 de la aplicación web eFuel}
\fecha{Septiembre~de~2018}
\agno{2018}
\fechadefensa{15~de~noviembre~de~2018}
\tutor{Tutor Académico: Prof. Soraya Carrasquel}
\usarcotutor
\cotutor{Tutor Industrial: Ing. José Cerqueiro}
\trabajo{Informe de Pasantía}
\coord{Ingeniería de la Computación}
\grado{Ingeniero en Computación}
\carrera{Ingeniería de la Computación}
\programa{Nombre del Programa}
\juradouno{Nombre y Apellido}
\juradodos{Nombre y Apellido}
\juradotres{Nombre y Apellido}

% Cambia comillas simple por comilla cerrada en ambiente verbatim
\makeatletter
\let \@sverbatim \@verbatim
\def \@verbatim {\@sverbatim \verbatimplus}
{\catcode`'=13 \gdef \verbatimplus{\catcode`'=13 \chardef '=13 }}
\makeatother

\begin{document}

\frontmatter
\maketitle
\begin{resumen}
     Es una exposici\'on clara del tema tratado en el trabajo, de los objetivos, de la metodolog\'ia utilizada, de los resultados relevantes obtenidos y de las conclusiones. Mismo tipo de fuente seleccionado con tamaño 12 e interlineado sencillo en el p\'arrafo. El resumen no debe exceder de trescientas (300) palabras escritas. \\
     Palabras cl\'aves: palabras, cl\'aves, separadas por coma, cinco m\'aximo.
\end{resumen}
\tableofcontents

\mainmatter
\chapter{Entorno empresarial}
En este capítulo se describe a la empresa en la cual se desarrolló el proyecto de pasantía. Comprende una breve reseña histórica, su misión y visión, la estructura organizacional y el área a la cual el pasante estuvo asignado.

\section{Antecedentes de la empresa}
IKêls Consulting se creó el año 2008 como una empresa dedicada al desarrollo de soluciones en el área de Sistemas de Información. Los fundadores contaban con una amplia trayectoria en los procesos y tecnología para la elaboración de documentación técnica avanzada (por ejemplo normas ISO para construcción de plantas petroquímicas).

Para aprovechar la experiencia previa los productos y servicios se concentran en el área de aplicaciones web (por ejemplo, sistemas de manejo de contenido o CMS) para el sector corporativo atendiendo a un selecto grupo de clientes con presencia local e internacional.

Actualmente, las actividades principales se concentran en:

\begin{itemize}
  \item Construcción de portales web en múltiples idiomas y que pueden ser administrados por sus propios dueños. Esto incluye la programación de módulos especiales para integrar información desde y hacia sistemas externos, desplegar datos de manera amigable o generar notificaciones automáticas dependientes de actividades de los visitantes u otros eventos.
  \item Apoyo en la gestión de contenido de portales web.
  \item Consultoría y gestión para optimizar las variables asociadas al rendimiento y desempeño de las páginas web. Teniendo especial interés en el monitoreo de presencia en buscadores, evaluación del perfil de los visitantes y garantizar un nivel adecuado de usabilidad en diferentes dispositivos, etc.
  \item Desarrollo de productos personalizados que complementen las ventajas y facilidades de los dispositivos móviles en sincronización con mecanismos de soporte en servidores web.
  \item Desarrollo de soluciones especializadas para ofrecer bajo el modelo SaaS o Software as a Service.
\end{itemize}

\section{Misión}
Proveer productos y servicios en el área de sistemas de información que permitan una comunicación efectiva de nuestros clientes con su público y también sirva como plataforma de trabajo donde se aprovechen las innovaciones y ventajas de las tecnologías más modernas.

\section{Visión}
Deseamos ser un proveedor confiable, que ofrece un alto valor agregado en cada producto o servicio que prestamos a nuestros clientes.

\section{Ubicación del pasante}
El proyecto de pasantía pertenece al grupo de desarrollo de aplicaciones y cuenta con la dirección del Presidente de la Empresa y con el apoyo de los ingenieros líderes del grupo.
\chapter{Marco Teórico}
En el presente capítulo se definen las bases teóricas sobre las cuáles se apoya el proyecto.

\section{Bases Teóricas}

\subsection{CMS}
Un CMS, por sus siglas en inglés \textit{Content Management System} (Sistema de Gestión de Contenidos), es una aplicación de software que provee algún nivel de automatización a las tareas de manejo de contenido. Un CMS permite a los usuarios crear nuevo contenido, editar contenido existente, y hacer el contenido accesible al público. \cite{cmsBarker}

Desde el punto de vista de un editor un CMS consta, básicamente, de 2 partes: una interfaz para la edición de contenido (referido como el \textit{back-end}, esto es, la capa de acceso a los datos de la aplicación) y una interfaz para mostrar el contenido publicado (referido como el \textit{front-end}, es decir, la capa de presentación de la aplicación).

El uso de un CMS facilita las tareas de mantenimiento y de generación de contenido, especialmente para usuarios que no tienen preparación técnica especial. Los usuarios de la aplicación eFuel serán, en su mayoría, personal sin conocimientos especializados en el área de computación, esta es una de las razones por las cuales resulta conveniente desarrollar el sistema sobre un CMS.

Además, el CMS sobre el cual se desarrolló la aplicación posee funcionalidades ya implementadas que son necesarias para el sistema: la autenticación de usuarios, permisología, interfaces para realizar operaciones sobre la base de datos, entre otras. La plataforma también cuenta con una variedad de librerías y de módulos que facilitan el desarrollo del sistema. También cabe destacar que la empresa tiene varios años de experiencia desarrollando aplicaciones web con un CMS lo cual facilitó el desarrollo del sistema.

\subsection{Modelo Cliente-Servidor}
Arquitectura de redes de computadoras que divide el trabajo entre 2 entidades: un cliente y un servidor. El cliente le envía una solicitud al servidor y espera una respuesta. Por otra parte, el servidor recibe la solicitud, lleva a cabo el trabajo requerido y devuelve una respuesta al cliente. \cite{redesTanenbaum} El servidor mantiene una relación de uno-a-muchos con los clientes. Es importante destacar que los términos “cliente” y “servidor” pueden referirse tanto a máquinas como a programas o procesos. Esta arquitectura es ampliamente utilizada en aplicaciones web.

\subsection{MVC}
MVC, siglas para Modelo-Vista-Controlador, es un modelo de diseño de aplicaciones compuesto por 3 partes: el Modelo (datos), la Vista (presentación de los datos e interfaz con el usuario) y el Controlador (proceso que maneja la entrada de los usuarios y el acceso a los datos). \cite{mvcKrasner} Al separar la aplicación en estos 3 componentes se reduce la complejidad del diseño arquitectónico y se incrementa la reusabilidad, flexibilidad y mantenimiento del código. Adicionalmente, se pueden realizar cambios sobre un componente sin afectar a los demás, lo cual permite que cada componente tenga ciclos de desarrollo independientes.

Actualmente, este patrón es ampliamente utilizado en el desarrollo de aplicaciones web ya que resulta muy natural acoplarlo con el modelo cliente-servidor que utiliza la web. El sistema eFuel fue desarrollado sobre una plataforma de desarrollo web que implementa el patrón MVC y, en consecuencia, el código tiene la estructura descrita por este patrón.

\subsection{API}
Un API, siglas en inglés para \textit{Application Programming Interface}, es un conjunto de comandos, funciones, protocolos y objetos que exponen los datos de una aplicación de software, es decir, establecen las reglas y los mecanismos a través de los cuales se puede tener acceso a ellos. \cite{apiChristensson}

Normalmente, aplicaciones externas disponen del API de una aplicación para obtener datos de esta última y usarlos para proveer algún servicio a sus usuarios.

\subsection{Servicio Web}
Un servicio web es un sistema de software diseñado para soportar interacción máquina-máquina a través de una red, proveen una vía estándar para la interoperabilidad entre distintas aplicaciones de software ejecutadas en distintas plataformas y ambientes. \cite{webServiceW3C} Típicamente, los sistemas externos tienen acceso al servicio web a través de un API.

El proyecto presente incluye el desarrollo de un módulo de servicios web para el acceso a algunos datos de la aplicación, por ejemplo, datos para generar tablas informativas.

\subsection{REST}
Transferencia de Estado Representacional o REST, por sus siglas en inglés, es un estilo de arquitectura para sistemas de hipermedia distribuidos (como la \textit{World Wide Web} o red informática mundial) que define una serie de restricciones que, cuando se aplican en conjunto, enfatizan la escalabilidad de interacciones entre componentes, la generalidad de las interfaces y el despliegue independiente de componentes. \cite{restFielding}

Las restricciones definidas para los sistemas REST son las siguientes:

\paragraph{Separación Cliente-Servidor} .
\paragraph{Sin estado (\emph{stateless} en inglés)} .
\paragraph{Permite el uso de memoria caché} .
\paragraph{Interfaz uniforme} cada solicitud al servidor debe tener los mismos componentes .
\paragraph{Sistema por capas} .

Se dice que un servicio web o el API de un servicio web es \textit{RESTful} cuando cumple con estas restricciones.
\chapter{Marco Tecnológico}
En este capítulo se definen las tecnologías y herramientas utilizadas para el desarrollo del sistema. Entre estas se incluyen los \textit{frameworks} utilizados para el desarrollo (ASP.NET, ASP.NET MVC, ASP.NET Web API y Umbraco), los lenguajes utilizados (C\# para el \textit{backend}, y HTML, CSS y JavaScript para el \textit{frontend}, como es común en el desarrollo web), la herramienta de control deversiones (Git) y el entorno de desarrollo (Visual Studio).

\section{Frameworks}
    \subsection{ASP.NET}
    ASP.NET es un modelo de desarrollo Web unificado que incluye los servicios necesarios para construir aplicaciones Web empresariales con un mínimo de codificación. ASP.NET es parte de .NET Framework y al codificar las aplicaciones en ASP.NET se tiene acceso a las clases de .NET Framework. \cite{asp.netMicrosoft}

    \subsection{ASP.NET MVC}
    ASP.NET MVC es un marco de trabajo (o \textit{framework} en inglés) para construir aplicaciones web escalables, basadas en estándares y usando patrones de diseño bien establecidos (el patrón MVC) y el poder de ASP.NET y .NET Framework. \cite{asp.netMVCMicrosoft}

    Se utilizó este marco de trabajo para el desarrollo del módulo \textbf{EF\_Core} (ver sección \emph{sección de arquitectura}) del proyecto, allí están implementados los controladores (ver sección \ref{mvc}) de la aplicación web.

    \subsection{ASP.NET Web Api}
    ASP.NET Web API es un marco de trabajo que facilita la construcción de servicios HTTP que llegan a una amplia gama de clientes, incluyendo navegadores y dispositivos móbiles. ASP.NET Web API es una plataforma ideal para construir aplicaciones RESTful en .NET Framework. \cite{asp.netWebAPIMicrosoft}

    Este marco de trabajo se utilizó para el desarrollo del módulo \textbf{EF\_API} (ver sección \emph{sección de arquitectura}) del proyecto, allí están implementados los servicios del API para acceder a información del sistema (ver sección \ref{webService}) de la aplicación web.

    \subsection{Umbraco}
    Umbraco es un Sistema de Gestión de Contenido gratuito y de código abierto construido sobre ASP.NET, fue desarrollado en Dinamarca y su primera versión fue lanzada en el año 2005. Umbraco cuenta con funcionalidades que son necesarias para el sistema: la autenticación de usuarios, permisología, interfaces para realizar operaciones sobre la base de datos, entre otras. La plataforma también cuenta con un repositorio de paquetes y extensiones que proveen otras funcionalidades útiles para la aplicación web y provee una variedad de librerías y de módulos que facilitan el desarrollo del sistema.

    Cabe destacar que la empresa tiene varios años de experiencia desarrollando aplicaciones web con Umbraco y varios de sus empleados tienen certificaciones y un nivel alto de conocimiento de la plataforma lo cual facilitó el desarrollo del sistema. La aplicación web eFuel fue desarrollada con Umbraco v7.10.

    \subsection{Bootstrap}


\section{Lenguajes}
A continuación se definen brevemente los lenguajes de programación utilizados para el desarrollo del proyecto divididos en \textit{front-end} y \textit{back-end}.

\subsection{\emph{Front-end}}
\subsubsection{HTML}
HTML (siglas en inglés para Lenguaje de Marcado de HiperTexto), es el lenguaje de marcado central de la \textit{World Wide Web} (red informática mundial). Originalmente, HTML fue diseñado principalmente para describir semánticamente documentos científicos. Sin embargo, la generalidad de su diseño ha permitido que sea adaptado, en los años siguientes, para describir otros tipos de documentos y hasta aplicaciones. \cite{htmlW3C}

Es el lenguaje que se usa para definir la estructura y el contenido de una página web, en el caso de este proyecto se usa para describir las vistas del sistema.

\subsubsection{CSS}
Hojas de Estilo en Cascada, o CSS por sus siglas en inglés (\textit{Cascading Style Sheets}), es un lenguaje de hojas de estilo que permite a los autores y usuarios adjuntar estilos (\textit{e.g.} fuentes y espaciados) a documentos estructurados (\textit{e.g.} documentos de HTML). Al separar el estilo de presentación del contenido de los documentos, se simplifica la autoría y el mantenimiento de los sitios. \cite{cssW3C}

\subsubsection{JavaScript}
JavaScript es un lenguaje de \textit{scripting} o programación que permite la creación de contenido dinámico, control de multimedia, animación de imágenes \cite{jsMozilla}. \emph{mencionar las llamadas al servidor para traer datos con el API}


\subsection{\emph{Back-end}}
\subsubsection{C\#}
C\# es un lenguaje con seguridad de tipos y orientado a objetos que permite a desarrolladores construir una variedad de aplicaciones robustas y seguras que corren sobre .NET Framework. \cite{cSharpMicrosoft}

\section{Control de versiones}
\subsection{Git}
Git es un sistema de gestión de versiones rápido, escalable y distribuido con un rico conjunto de comandos que proveen operaciones de alto nivel y acceso completo a internos. \cite{gitGit} Fue desarrollado por Linus Torvald en el año 2005 para facilitar el trabajo de varios desarrolladores sobre un mismo proyecto. Permite llevar el seguimiento de los cambios a un grupo de archivos y sincronizar varios repositorios en máquinas distintas.

Este es el sistema de gestión de versiones usado en la empresa en la que se desarrolló el proyecto.

\section{Entorno de trabajo}
\subsection{Visual Studio}
Visual Studio es un ambiente de desarrollo integrado que permite editar, depurar, compilar y publicar código. \cite{visualStudioMicrosoft} Este ambiente de desarrollo incluye una gran cantidad de funcionalidades para facilitar el desarrollo de software: depuración del código paso a paso y con información detallada de las variable y otras entidades del programa, instrucciones de compilación complejas para la aplicación, descarga y actualización paquetes y librerías, integración con Git, \textit{IntelliSense} de Microsoft para la completación de partes de código usando el contexto de la aplicación (clases y sus relaciones y métodos), entre otras.

Este entorno de trabajo está muy bien integrado con los \textit{frameworks} utilizados para el proyecto (ASP.NET y sus extensiones/derivados), por lo que su uso resultó ventajoso y natural para el desarrollo del proyecto.

\subsection{Trello}
Trello es una herramienta de colaboración que organiza las tareas de un proyecto en tablas. Trello informa en qué se está trabajando, quién está trabajando en qué, y el estado de una tarea en el proceso de desarrollo.

Cada tabla tiene varias listas y cada lista contiene tarjetas. Una tabla corresponde a un proyecto, una lista corresponde al estado de una tarea y una tarjeta a una tarea, por ejemplo, en una tabla suele haber 3 listas: \emph{Por hacer}, \emph{Haciendo} y \emph{Lista}. Las tarjetas contienen una descripción de la tarea, la o las personas asignadas y puede tener una fecha límite de entrega. A medida que se va completando trabajo se van moviendo las tarjetas de una lista a otra.

Resultó una herramienta útil para llevar el control de las tareas realizadas y por realizar. Se usó esta herramienta como apoyo para el uso de Scrum (ver capítulo \ref{marcoMetodologico}).
\nocite{*}
\bibliography{referencias}
%\appendix

\end{document}
