\documentclass{article}

\usepackage{fancyhdr}
\usepackage{ragged2e}
\usepackage[table]{xcolor}
\usepackage{indentfirst}

\makeatletter
\title{Documento de Arquitectura del Software} \let\Title\@title
\date{Septiembre 2018} \let\Date\@date
\author{Ricardo Münch} \let\Author\@author
\makeatother

\pagestyle{fancy}
\fancyhf{}
\rhead{Versión: 1.0 \\ \Date}
\lhead{eFuel \\ DAS \\ Identificador del documento}

\lfoot{Confidencial}
\cfoot{iKels Consulting \\ \Date}
\rfoot{Pag. \thepage}

\renewcommand{\headrulewidth}{1pt}
\renewcommand{\footrulewidth}{1pt}
\renewcommand{\contentsname}{Índice}

\begin{document}
    \begin{titlepage}
    \huge{\Title}
    \begin{flushright}
        \Large{eFuel \\ Versión: 1.0}
    \end{flushright}
    \end{titlepage}

    \newpage
    \tableofcontents

    \newpage
    \begin{center}
        \begin{tabular}{ |c|c|c|c| } 
            \hline
            \rowcolor{lightgray}
            Fecha & Versión & Descripción & Autores \\
            \hline 
            29/09/2018 & 1.0 & Documento de Arquitectura & Ricardo Münch \\
            \hline
        \end{tabular}
    \end{center}

    \newpage
    \section{Introducción}
    El objetivo principal de la arquitectura del software es aportar conceptos y un lenguaje común que ayuden a describir el software y permita la comunicación entre el cliente y los diseñadores.
    
    \subsection{Propósito}
    Este documento busca hacer una abstracción de lo que será el sistema a través de algunas vistas de la arquitectura del mismo. Se pretende definir algunos elementos estructurales que describen el sistema \emph{eFuel}.
    
    \subsection{Alcance}
    A continuación presentamos una abstracción de la estructura que debe tener el sistema. El documento contempla la vista lógica, la vista de datos y las características no funcionales que debe tener el sistema.
        
    \subsection{Referencias}
    No hace referencia a ningún otro documento.
    
    \subsection{Vista Global}
    Este documento comprende 6 secciones en las cuales se elaboran los distintos aspectos de la arquitectura de \emph{eFuel}, tanto a nivel de software como de hardware. En la sección \ref{reprArq} se introduce la representación arquitectónica del sistema. Luego, en la sección \ref{metasArq}, se enumeran los objetivos y restricciones que suscriben la arquitectura presentada. Seguidamente, se describen las distintas vistas que conforman la arquitectura de la sección \ref{vistaCasosDeUso} a la \ref{vistaDatos}, teniendo en la sección \ref{vistaCasosDeUso} la Vista de Casos de Uso. Finalmente, la sección \ref{tamDesemp}.


    \section{Representación Arquitectónica} \label{reprArq}
    La representación arquitectónica de \emph{eFuel} está basada en el modelo de 4+1 vistas de Philippe Kruchten. En el transcurso del documento se tratarán más a fondo los detalles de cada una.

    \section{Metas y Restricciones Arquitectónicas} \label{metasArq}

    
    \section{Vista de Casos de Uso} \label{vistaCasosDeUso}
    En esta vista se describirá el sistema desde el punto de vista de los casos de uso.
    \subsection{Resumen de Casos de Uso}
    \begin{center}
        \begin{tabular}{ |c|c|c| } 
            \hline
            \rowcolor{lightgray}
            ID Caso de Uso & Caso de Uso & Actor \\
            \hline
            CU- & Gestionar clientes & Staff \\
            CU- & Agregar cliente & Staff \\
            CU- & Consultar lista de clientes & Staff, Customer \\
            CU- & Consultar detalles de cliente & Staff, Customer \\
            CU- & Editar cliente & Staff \\
            CU- & Eliminar de cliente/s & Staff \\
            CU- & Importar lista de clientes & Staff, Customer \\
            CU- & Exportar lista de clientes & Staff, Customer \\

            CU- & Gestionar transportes & Staff \\
            CU- & Agregar transporte & Staff \\
            CU- & Consultar lista de transportes & Staff, Customer \\
            CU- & Consultar detalles de transporte & Staff, Customer \\
            CU- & Editar transporte & Staff \\
            CU- & Eliminar de transporte/s & Staff \\
            CU- & Importar lista de transportes & Staff, Customer \\
            CU- & Exportar lista de transportes & Staff, Customer \\

            CU- & Gestionar zonas & Staff \\
            CU- & Agregar zona & Staff \\
            CU- & Consultar lista de zonas & Staff, Customer \\
            CU- & Consultar detalles de zona & Staff, Customer \\
            CU- & Editar zona & Staff \\
            CU- & Eliminar de zona/s & Staff \\
            CU- & Importar lista de zonas & Staff, Customer \\
            CU- & Exportar lista de zonas & Staff, Customer \\

            CU- & Gestionar turnos & Staff \\
            CU- & Agregar turno & Staff \\
            CU- & Consultar lista de turnos & Staff \\
            CU- & Consultar detalles de turno & Staff \\
            CU- & Editar turnos & Staff \\
            CU- & Eliminar de turno/s & Staff \\
            \hline
        \end{tabular}
    \end{center}

    \subsection{Diagrama de Casos de Uso}
    \subsection{Especificaciones de Casos de Uso}

    \section{Vista Lógica} \label{vistaLogica}
    \subsection{Vista General}
    \subsubsection{Diagrama Conceptual (Modelo de Dominio)}
    \subsubsection{Diagrama de Clases}

    \section{Vista de Implantación} \label{vistaImplantacion}
    \subsection{Configuración Estándar}
    \subsection{Diagrama de Despliegue}


    \section{Vista de Implementación} \label{vistaImplementacion}
    \subsection{Vista General}
    \subsection{Diagrama de Componentes}


    \section{Vista de Datos} \label{vistaDatos}
    \subsection{Diagrama de Entidad Relación (ER)}
    \subsection{Diccionario de Datos}


    \section{Tamaño y Desempeño} \label{tamDesemp}

    \section{Calidad}
    \subsection{Mantenibilidad}
    \subsection{Flexibilidad}
    \subsection{Seguridad}
\end{document}

