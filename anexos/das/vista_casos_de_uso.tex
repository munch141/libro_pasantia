    \section{Vista de Casos de Uso} \label{vistaCasosDeUso}
    En esta vista se describirá el sistema desde el punto de vista de los casos de uso. El sistema tiene 3 actores:

    \begin{itemize}
        \item \emph{Admin}: administrador de eFuel. Es un usuario de Umbraco, esto es, cuenta con las credenciales para ingresar al back end de Umbraco y, además, tiene los permisos necesarios para administrar las entidades y los miembros de eFuel.
        \item \emph{eFuel Member}: miembro de eFuel, posee credenciales para entrar al sistema.
        \item \emph{Customer}: representa a una o varias estaciones de servicio. Solo tiene acceso a la información referente a las estaciones de servicio asignadas por el administrador del sistema. Es el tipo de miembro que más uso le dará al sistema.
        \item \emph{Staff}: representa a un distribuidor de combustible. Tiene acceso a la información de todas las estaciones de servicio del sistema.
    \end{itemize}

    \subsection{Resumen de Casos de Uso}

    \newcounter{magicrownumbers}
    \newcommand\rownumber{\stepcounter{magicrownumbers}\arabic{magicrownumbers}}

    \begin{longtable}{ | l | l | c | }
        \hline
        \rowcolor{gray!30}
        \multicolumn{1}{|c|}{ID del Caso de Uso} &
        \multicolumn{1}{|c|}{Caso de Uso} &
        \multicolumn{1}{|c|}{Actor} \\
        \hhline{===}
        \endhead

        CU-\rownumber & Iniciar sesión (Umbraco) & Admin \\ \hline
        CU-\rownumber & Consultar lista de miembros & Admin \\ \hline
        CU-\rownumber & Gestionar miembro (CRUD) & Admin \\ \hline
        CU-\rownumber & Asignar cliente/s a miembro & Admin \\ \hline
        CU-\rownumber & Remover cliente/s de miembro & Admin \\ \hline
        CU-\rownumber & Cambiar permisos de miembro & Admin \\ \hline

        CU-\rownumber & Gestionar contenido & Admin \\ \hline
        CU-\rownumber & Consultar lista de clientes & Admin \\ \hline
        CU-\rownumber & Gestionar cliente (CRUD) & Admin \\ \hline
        CU-\rownumber & Consultar lista de productos & Admin \\ \hline
        CU-\rownumber & Gestionar producto (CRUD) & Admin \\ \hline
        CU-\rownumber & Consultar lista de transportes & Admin \\ \hline
        CU-\rownumber & Gestionar transportes (CRUD) & Admin \\ \hline
        CU-\rownumber & Consultar lista de zonas & Admin \\ \hline
        CU-\rownumber & Gestionar zonas (CRUD) & Admin \\ \hline

        CU-\rownumber & Gestionar transacciones & Admin \\ \hline
        CU-\rownumber & Consultar lista de registros & Admin \\ \hline
        CU-\rownumber & Gestionar registro (CRUD) & Admin \\ \hline
        CU-\rownumber & Consultar lista de pedidos & Admin \\ \hline
        CU-\rownumber & Gestionar pedido (CRUD) & Admin \\ \hline
        CU-\rownumber & Consultar lista de detalles de pedidos & Admin \\ \hline
        CU-\rownumber & Gestionar detalle de pedido (CRUD) & Admin \\ \hline
        CU-\rownumber & Consultar lista de facturas & Admin \\ \hline
        CU-\rownumber & Gestionar factura (CRUD) & Admin \\ \hline
        CU-\rownumber & Importar lista de facturas & Staff \\
        \hline
        CU-\rownumber & Consultar lista de cobros & Admin \\ \hline
        CU-\rownumber & Gestionar cobro (CRUD) & Admin \\ \hline
        CU-\rownumber & Consultar lista de detalles de cobros & Admin \\ \hline
        CU-\rownumber & Gestionar detalle de cobro (CRUD) & Admin \\ \hline

        CU-\rownumber & Iniciar sesión (eFuel) & Customer, Staff \\ \hline
        CU-\rownumber & Consultar lista de pedidos & Customer, Staff \\ \hline
        CU-\rownumber & Consultar pedido & Customer, Staff \\ \hline
        CU-\rownumber & Filtrar lista de pedidos & Customer, Staff \\ \hline
        CU-\rownumber & Exportar lista de pedidos & Customer, Staff \\ \hline
        CU-\rownumber & Crear pedido & Customer, Staff \\ \hline
        CU-\rownumber & Seleccionar cliente & Customer, Staff \\ \hline
        CU-\rownumber & Seleccionar fecha & Customer, Staff \\ \hline
        CU-\rownumber & Seleccionar turno  & Customer, Staff \\ \hline
        CU-\rownumber & Seleccionar transporte  & Customer, Staff \\ \hline
        CU-\rownumber & Seleccionar productos & Customer, Staff \\ \hline

        CU-\rownumber & Consultar lista de facturas & Customer, Staff \\ \hline

        CU-\rownumber & Consultar lista de clientes & Customer, Staff \\ \hline
        CU-\rownumber & Consultar cliente & Customer, Staff \\ \hline
        CU-\rownumber & Consultar pedidos de cliente & Customer, Staff \\ \hline
        CU-\rownumber & Importar lista de clientes & Staff \\
        \hline

        CU-\rownumber & Consultar lista de transportes & Customer, Staff \\ \hline

        CU-\rownumber & Importar lista de transportes & Staff \\ \hline

        CU-\rownumber & Consultar lista de zonas & Customer, Staff \\ \hline

        CU-\rownumber & Importar lista de zonas & Staff \\
        \hline

        CU-\rownumber & Importar lista de despachos & Staff \\
        \hline

        CU-\rownumber & Reportar pagos & Customer, Staff \\
        \hline

        \caption{Resumen de casos de uso}
        \label{table:cu}
    \end{longtable}

    \subsection{Diagrama de Casos de Uso}

    \begin{figure}[H]
        \includegraphics[width=.9\textwidth,height=.9\textheight,keepaspectratio]{cu_admin.png}
        \caption{Casos de Uso back end (\emph{admin})}
        \label{fig:cu_admin}
        \centering
    \end{figure}

    \begin{figure}[H]
        \includegraphics[width=.9\textwidth,height=.9\textheight,keepaspectratio]{cu_customer_staff.png}
        \caption{Casos de Uso front end (\emph{customer} y \emph{staff})}
        \label{fig:cu_customer_staff}
        \centering
    \end{figure}

    \subsection{Especificaciones de Casos de Uso}
    A continuación las narrativas de los casos de uso:

    \begin{center}
        \begin{longtabu} to 0.9\textwidth { | X[p] | X[p] | }
            \hline
            \multicolumn{2}{|l|}{
                \cellcolor{gray!30}{\large{\textbf{Caso de Uso:}} Iniciar sesión (Umbraco)}
            } \TBstrut \\
            \hline\hline

            \multicolumn{2}{|l|}{
                \makecell{\large{\textbf{Descripción:}} \\ El usuario quiere ingresar al back end de Umbraco.}
            } \\
            \hline

            \multicolumn{2}{|l|}{
                \makecell{\large{\textbf{Precondición:}} \\ Haber ingresado la dirección correcta del sitio en la barra de navegación.}
            } \\
            \hline


            \multicolumn{2}{|l|}{\cellcolor{gray!15}\large{\textbf{Flujo básico:}}}  \TBstrut\\
            \hline

            Actor & Sistema \TBstrut\\
            \hline
            1. El actor abre su navegador e introduce la dirección correspondiente al back end de Umbraco. &  \\ [0.3ex]
            \hline
             & 2. El servidor procesa la solicitud y envía al navegador del cliente una ventana para que el usuario se autentique. \\ [0.3ex]
             \hline
             3. El actor introduce su email y su contraseña. &  \\ [0.3ex]
             \hline
             & 4. El sistema valida la información del usuario y lo redirige al tablero (back end) de Umbraco. \\ [0.3ex]
             \hline\hline


            \multicolumn{2}{|l|}{\cellcolor{gray!15}\large{\textbf{Flujos alternos:}}}  \TBstrut\\
            \hline

            Actor & Sistema \TBstrut\\
            \hline
            1. El actor abre su navegador e introduce la dirección correspondiente al back end de Umbraco. &  \\ [0.3ex]
            \hline
             & 2. El servidor procesa la solicitud y envía al navegador del cliente una ventana para que el usuario se autentique. \\ [0.3ex]
             \hline
             3. El actor introduce su email y su contraseña. &  \\ [0.3ex]
             \hline
             & 4. El sistema no valida la información del usuario y lo redirige a la misma página de inicio de sesión indicándole que los datos introducidos son incorrectos. \\ [0.3ex]
             \hline\hline

            \multicolumn{2}{|l|}{
                \makecell{\large{\textbf{Poscondición:}} \\ El usuario se encuentra en el tablero de Umbraco.}
            } \\
            \hline
            \multicolumn{2}{|l|}{
                \makecell{\large{\textbf{Puntos de extensión:}} \\ No se requiere de otros casos de uso.}
            } \\
            \hline
        \end{longtabu}
    \end{center}
    \vspace{-4em}