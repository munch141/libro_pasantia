    \section{Vista de Casos de Uso} \label{vistaCasosDeUso}
    En esta vista se describirá el sistema desde el punto de vista de los casos de uso. El sistema tiene 3 actores:

    \begin{itemize}
        \item \emph{Customer}: representa a una o varias estaciones de servicio. Solo tiene acceso a la información referente a las estaciones de servicio asignadas por el administrador del sistema. Es el tipo de usuario que más uso le dará al sistema.
        \item \emph{Staff}: representa a un distribuidor de combustible. Tiene acceso a la información de todas las estaciones de servicio del sistema.
        \item \emph{Admin}: administrador del sistema. Tiene acceso al \emph{back end} de Umbraco y puede gestionar (realizar las acciones CRUD) todas las entidades del sistema, incluyendo a los usuarios.
    \end{itemize}

    \subsection{Resumen de Casos de Uso}
    \newcounter{magicrownumbers}
    \newcommand\rownumber{\stepcounter{magicrownumbers}\arabic{magicrownumbers}}
    \begin{center}
        \begin{longtable}{ | l | l | c | }
            \hline
            \rowcolor{gray!30}
            \multicolumn{1}{|c|}{ID del Caso de Uso} &
            \multicolumn{1}{|c|}{Caso de Uso} &
            \multicolumn{1}{|c|}{Actor} \\
            \hhline{===}
            \endhead

            \endfoot

            CU-\rownumber & Gestionar usuarios & Admin \\ \hline
            CU-\rownumber & Consultar lista de usuarios & Admin \\ \hline
            CU-\rownumber & Gestionar usuario (CRUD) & Admin \\ \hline
            CU-\rownumber & Asignar cliente/s a usuario & Admin \\ \hline
            CU-\rownumber & Remover cliente/s de usuario & Admin \\ \hline
            CU-\rownumber & Cambiar permisos de usuario & Admin \\ \hline

            CU-\rownumber & Gestionar contenido & Admin \\ \hline
            CU-\rownumber & Consultar lista de clientes & Admin \\ \hline
            CU-\rownumber & Gestionar cliente (CRUD) & Admin \\ \hline
            CU-\rownumber & Consultar lista de productos & Admin \\ \hline
            CU-\rownumber & Gestionar producto (CRUD) & Admin \\ \hline
            CU-\rownumber & Consultar lista de transportes & Admin \\ \hline
            CU-\rownumber & Gestionar transportes (CRUD) & Admin \\ \hline
            CU-\rownumber & Consultar lista de zonas & Admin \\ \hline
            CU-\rownumber & Gestionar zonas (CRUD) & Admin \\ \hline

            CU-\rownumber & Gestionar transacciones & Admin \\ \hline
            CU-\rownumber & Consultar lista de registros & Admin \\ \hline
            CU-\rownumber & Gestionar registro (CRUD) & Admin \\ \hline
            CU-\rownumber & Consultar lista de pedidos & Admin \\ \hline
            CU-\rownumber & Gestionar pedido (CRUD) & Admin \\ \hline
            CU-\rownumber & Consultar lista de detalles de pedidos & Admin \\ \hline
            CU-\rownumber & Gestionar detalle de pedido (CRUD) & Admin \\ \hline
            CU-\rownumber & Consultar lista de facturas & Admin \\ \hline
            CU-\rownumber & Gestionar factura (CRUD) & Admin \\ \hline
            CU-\rownumber & Consultar lista de cobros & Admin \\ \hline
            CU-\rownumber & Gestionar cobro (CRUD) & Admin \\ \hline
            CU-\rownumber & Consultar lista de detalles de cobros & Admin \\ \hline
            CU-\rownumber & Gestionar detalle de cobro (CRUD) & Admin \\ \hline

            CU-\rownumber & Iniciar sesión & Customer, Staff \\ \hline
            CU-\rownumber & Consultar lista de pedidos & Customer, Staff \\ \hline
            CU-\rownumber & Consultar pedido & Customer, Staff \\ \hline
            CU-\rownumber & Filtrar lista de pedidos & Customer, Staff \\ \hline
            CU-\rownumber & Exportar lista de pedidos & Customer, Staff \\ \hline
            CU-\rownumber & Crear pedido & Customer, Staff \\ \hline
            CU-\rownumber & Seleccionar cliente & Customer, Staff \\ \hline
            CU-\rownumber & Seleccionar fecha & Customer, Staff \\ \hline
            CU-\rownumber & Seleccionar turno  & Customer, Staff \\ \hline
            CU-\rownumber & Seleccionar transporte  & Customer, Staff \\ \hline
            CU-\rownumber & Seleccionar productos & Customer, Staff \\ \hline

            CU-\rownumber & Consultar lista de facturas & Customer, Staff \\ \hline

            CU-\rownumber & Consultar lista de clientes & Customer, Staff \\ \hline
            CU-\rownumber & Consultar cliente & Customer, Staff \\ \hline
            CU-\rownumber & Consultar pedidos de cliente & Customer, Staff \\ \hline

            CU-\rownumber & Consultar lista de transportes & Customer, Staff \\ \hline

            CU-\rownumber & Importar lista de transportes & Customer, Staff \\ \hline

            CU-\rownumber & Consultar lista de zonas & Customer, Staff \\ \hline

            CU-\rownumber & Importar lista de zonas & Customer, Staff \\ \hline

        \end{longtable}
    \end{center}

    \subsection{Diagrama de Casos de Uso}
    Se separaron los casos de uso en varios diagramas para facilitar la lectura.

    \begin{figure}[H]
        \includegraphics[width=\textwidth]{cu_admin.png}
        \centering
    \end{figure}

    \begin{figure}[H]
        \includegraphics[width=\textwidth]{cu_customer_staff.png}
        \centering
    \end{figure}

    \subsection{Especificaciones de Casos de Uso}
    A continuación las narrativas de los casos de uso:

    \begin{center}
        \begin{longtabu} to 0.9\textwidth { | X[p] | X[p] | }
            \hline
            \multicolumn{2}{|l|}{
                \cellcolor{gray!30}{\large{\textbf{Caso de Uso:}} Gestionar usuarios}
            } \TBstrut \\
            \hline\hline

            \multicolumn{2}{|l|}{
                \makecell{\large{\textbf{Descripción:}} \\ Muestra en pantalla el módulo de gestión de miembros.}
            } \\
            \hline

            \multicolumn{2}{|l|}{
                \makecell{\large{\textbf{Precondición:}} \\ Haber ingresado la dirección correcta del sitio en la barra de navegación.}
            } \\
            \hline

            
            \multicolumn{2}{|l|}{\cellcolor{gray!15}\large{\textbf{Flujo básico:}}}  \TBstrut\\
            \hline

            Actor & Sistema \TBstrut\\
            \hline

            1. El actor abre su navegador e introduce la dirección correspondiente al back end de Umbraco. &  \\
            \hline
             & 2. El servidor procesa la solicitud y envía al navegador del cliente una ventana para que el usuario se autentique. \\
             \hline\hline


            \multicolumn{2}{|l|}{\cellcolor{gray!15}\large{\textbf{Flujos alternos:}}}  \TBstrut\\
            \hline
            Actor & Sistema \TBstrut\\
            \hline
            1. El actor hace algo. &  \TBstrut\\
            \hline
             & 2. El sistema responde. \TBstrut\\
             \hline\hline

            \multicolumn{2}{|l|}{
                \makecell{\large{\textbf{Poscondición:}} \\ Aquí va la poscondición del caso de uso}
            } \\
            \hline
            \multicolumn{2}{|l|}{
                \makecell{\large{\textbf{Requerimientos especiales:}} \\ Aquí van los requerimientos especiales del caso de uso}
            } \\
            \hline
            \multicolumn{2}{|l|}{
                \makecell{\large{\textbf{Puntos de extensión:}} \\ Aquí van los puntos de extensión del caso de uso}
            } \\
            \hline
        \end{longtabu}
    \end{center}

    \begin{center}
        \begin{longtabu} to 0.9\textwidth { | X[p] | X[p] | }
            \hline
            \multicolumn{2}{|l|}{
                \cellcolor{gray!30}{\large{\textbf{Caso de Uso:}} Consultar lista de usuarios}
            } \TBstrut \\
            \hline\hline

            \multicolumn{2}{|l|}{
                \makecell{\large{\textbf{Descripción:}} \\ Muestra en pantalla una lista con todos los miembros del sistema.}
            } \\
            \hline

            \multicolumn{2}{|l|}{
                \makecell{\large{\textbf{Precondición:}} \\ Haber ingresado exitosamente al back end de Umbraco.}
            } \\
            \hline

            
            \multicolumn{2}{|l|}{\cellcolor{gray!15}\large{\textbf{Flujo básico:}}}  \TBstrut\\
            \hline

            Actor & Sistema \TBstrut\\
            \hline

            1. El actor hace click en la pestaña de \emph{Members} en el back end de Umbraco. & \\
            \hline
             & 2. El sistema muestra una vista con la lista de los miembros. \\
             \hline\hline


            \multicolumn{2}{|l|}{\cellcolor{gray!15}\large{\textbf{Flujos alternos:}}}  \TBstrut\\
            \hline
            Actor & Sistema \TBstrut\\
            \hline
            1. El actor hace algo. &  \TBstrut\\
            \hline
             & 2. El sistema responde. \TBstrut\\
             \hline\hline

            \multicolumn{2}{|l|}{
                \makecell{\large{\textbf{Poscondición:}} \\ Aquí va la poscondición del caso de uso}
            } \\
            \hline
            \multicolumn{2}{|l|}{
                \makecell{\large{\textbf{Requerimientos especiales:}} \\ Aquí van los requerimientos especiales del caso de uso}
            } \\
            \hline
            \multicolumn{2}{|l|}{
                \makecell{\large{\textbf{Puntos de extensión:}} \\ Aquí van los puntos de extensión del caso de uso}
            } \\
            \hline
        \end{longtabu}
    \end{center}



