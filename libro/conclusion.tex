\chapter*{Conclusiones y Recomendaciones}
Las conclusiones a continuación resumen los resultados y hallazgos más relevantes obtenidos durante el proyecto de pasantía:

\begin{itemize}
    \item Se desarrolló una versión funcional de la aplicación web eFuel que abarca el 92\% de los casos de uso estipulados. Este alcance fue acordado por la empresa y el pasante, la empresa quedó satisfecha con el trabajo realizado.
    \item La aplicación funciona como un agregado a un sitio de Umbraco genérico, es decir, si se tiene un sitio un sitio de Umbraco montado en un servidor se puede instalar eFuel junto con este sitio sin interferir con el funcionamiento de este.
    \item Durante el desarrollo hubo algunos retrasos y refactorizaciones en el código (de hecho no se pudo completar la funcionalidad del módulo de conciliación de pagos), esto se debe a 2 factores: la falta de documentación formal al inicio del proyectoy a algunas fallas de comunicación por parte del duaño del producto y del pasante. Sabiendo esto, hay que resaltar la importancia de generar documentos que ayuden a plasmar una visión del sistema para que todos los actores tengan claro qué hace el sistema y cómo se espera que lo haga, de esta manera el desarrollo será más fluido.
    \item Al enfrentarse con nuevas tecnologías y conceptos es importante saber consultar tanto fuentes de información (documentación, libros, etc) como al equipo de trabajo, ya que muchas veces otra persona del equipo ha resolvido problemas similares o tiene experiencia con las tecnologías utilizadas y puede ayudar a avanzar con el desarrollo de una manera importante. Hay que saber cómo y dónde buscar información.
\end{itemize}

Como recomendaciones se tiene lo siguiente:
\begin{itemize}
    \item Tener documentación adecuada y actualizada del sistema desde el inicio, esto facilita el desarrollo y provee claridad para todos involucrados con el producto.
    \item Realizar pruebas de todo tipo sobre la aplicación antes de continuar con el desarrollo, ya que no se realizaron pruebas durante la pasantía.
    \item Elaborar una guía de usuario.
    \item Proveer más opciones de gestión de datos en el front end de la aplicación para que el usuario no deba alternar entre la interfaz de Umbraco y el front end de eFuel.
    \item Tomar pasos para que la aplicación sea más fácil de instalar en un sitio de Umbraco, apuntar a hacer la aplicación un paquete de Umbraco.
\end{itemize}

El pasante obtuvo muchas enseñanzas tanto a nivel tecnológico como a nivel personal, sobre todo con respecto a lo que implica el proceso de desarrollo de software desde la concepción hasta la entrega del producto. La documentación y la planificación son integrales para poder crear un producto de buena calidad y en tiempos deseables, se debe tener siempre en mente que muchos componentes del software cambian a lo largo del desarrollo, también es importante saber cuando se debe pedir ayuda a otras personas del equipo para poder agilizar el desarrollo de la mejor manera posible.

El proyecto de pasantía fue una excelente oportunidad para poner en práctica muchos conceptos y teorías aprendidas en la carrera, para reforzar y complementarlos, para entender por qué son importantes pero también que la teoría y la práctica pueden ser muy diferentes y es el trabajo del ingeniero reconciliar las 2 para mejorar siempre los procesos de desarrollo y sus resultados.