\chapter{Marco Metodológico} \label{marcoMetodologico}

En este capítulo se describirá el marco de desarrollo Scrum que fue utilizado como metodología para el desarrollo del sistema, se dará una definición de Scrum y luego se definirán los roles de un equipo Scrum, los artefactos utilizados y los eventos que deben ocurrir durante el desarrollo.

\section{Definición de Scrum}
Scrum es un marco de trabajo dentro del cual la gente puede abordar un problema adaptativo complejo, mientras se entregan productos del mayor valor posible. Es un marco de trabajo para procesos que ha sido usado para gestionar el trabajo en productos complejos desde los inicios de la década de los 90. Scrum no es un proceso, ni una técnica o método definitivo. Más bien, es un marco de trabajo dentro del cual se pueden emplear varias técnicas y procesos. El marco de Scrum consiste en Equipos Scrum y sus roles, eventos, artefactos y reglas asociados. \cite{scrumSchwaber}

\subsection{El equipo de Scrum}
El equipo de Scrum está formado por el Dueño del Producto, el Equipo de Desarrollo y un \emph{Scrum Master}. Los equipos de Scrum son auto-organizados y auto-suficientes. Los equipos auto-organizados escogen la mejor forma de llevar a cabo el trabajo, en vez de ser dirigidos por otras personas fuera del equipo. Los equipos auto-suficientes tienen todas las competencias necesarias para terminar el trabajo sin depender de otros fuera del equipo. \cite{scrumSchwaber} A continuación se describirán los roles de un equipo Scrum:

\subsubsection{Dueño del Producto} \label{productOwner}
Es el responsable de maxizimizar el valor del producto desarrollado por el Equipo de Desarrollo. Es el responsable de manejar la Lista de Objetivos del Producto (ver sección \ref{productBacklog}), esto incluye:

\begin{itemize}
    \item Expresar los items de la Lista de Objetivos del Producto.
    \item Ordenar los items de la Lista de Objetivos para lograr los objetivos de la mejor manera posible.
    \item Optimizar el valor del trabajo realizado por el Equipo de Desarrollo (ver sección \ref{developmentTeam}).
    \item Asegurarse de que la Lista de Objetivos del Producto sea visible, transparente y clara para todos.
    \item Asegurarse de que el Equipo de Desarrollo entienda los items a un nivel adecuado.
\end{itemize}

El Dueño del Producto puede realizar el trabajo descrito arriba o puede delegarlo al Equipo de Desarrollo, sin embargo, siempre será responsable de este.

Para que el Dueño del Producto sea exitoso, toda la organización debe respetar sus decisiones. Estas decisiones son visibles en el contenido y el ordenamiento de la Lista de Objetivos del Producto. \cite{scrumSchwaber}

\subsubsection{Equipo de Desarrollo} \label{developmentTeam}
El Equipo de Desarrollo es un grupo de profesionales que realizan el trabajo para entregar un Incremento (ver sección \ref{increment}) listo del producto para cada Iteración (ver sección \ref{sprint}). Los Equipos de Desarrollo están estructurados y tienen la potestad de gestionar su propio trabajo. Los Equipos de Desarrollo tienen las siguientes características:

\begin{itemize}
    \item Están auto-organizados. Nadie le dice al equipo cómo debe llevar a cabo las tareas de la Lista de Objetivos del Producto.
    \item Son auto-suficientes, es decir, tienen todas las habilidades necesarias para llevar crear un Incremento.
    \item Scrum no tiene títulos para las personas que conforman el equipo, independientemente del trabajo que haga cada persona.
    \item Scrum no reconoce sub-equipos dentro del un Equipo de Desarrollo.
    \item Los miembros del equipo pueden tener habilidades específicas, pero la responsabilidad pertenece al equipo con un todo. \cite{scrumSchwaber}
\end{itemize}

\subsubsection{Facilitador} \label{scrumMaster}
El Facilitador (o \emph{Scrum Master}) es el responsable de promover y apoyar Scrum como está definido en la Guía de Scrum (ver \cite{scrumSchwaber}). Esto lo logran ayudando a todo el Equipo Scrum a entender la teoría, práctica, reglas y valores de Scrum. \cite{scrumSchwaber}

\subsection{Eventos de Scrum}
\subsubsection{La Iteración} \label{sprint}


\subsection{Artefactos de Scrum}
\subsubsection{Lista de Objetivos del Producto} \label{productBacklog}
La Lista de Objetivos del Producto es una lista ordenada de todas las tareas que deben ser completadas para entregar el producto. Esta lista nunca está completa, la primera versión muestra los primeros requerimientos conocidos y mejor entendidos. \cite{scrumSchwaber} La Lista de Objetivos del Producto evoluciona junto con el producto para actualizar los requerimientos dependiendo de las exigencias del cliente y del mercado. El Dueño del Producto (ver sección \ref{productOwner}) es el responsable de este artefacto.

\subsubsection{Lista de Tareas de la Iteración}
La Lista de Tareas de la Iteración es el conjunto de tareas de la lista de objetivos del producto seleccionadas para completar en una Iteración, puede verse como una predicción de lo que estará completado al llevarse a cabo la Iteración. \cite{scrumSchwaber} Esta lista va a ir cambiando a medida que el Equipo de Desarrollo identifique tareas y trabajo que debe ser realizado para completar los items durante el desarrollo de la Iteración, es decir, además de las tareas tomadas de la Lista de Objetivos del Producto, pueden surgir nuevas tareas que deben ser cumplidas para llevar a cabo el trabajo de la Iteración.

\subsubsection{Tablero de tareas}
El tablero de tareas es una tabla utilizada para gestionar el estado de los objetivos de la lista de objetivos del producto, contiene 4 columnas:

\begin{itemize}
    \item \textbf{Por hacer}: contiene las tareas que no se han empezado.
    \item \textbf{Haciendo}: contiene las tareas en las que se está trabajando actualmente.
    \item \textbf{Hechas}: contiene las tareas que se han completado.
    \item \textbf{Mejoras}: contiene tareas u objetivos que salen del alcance del proyecto actual y que pueden ser desarrolladas para una versión futura del producto.
\end{itemize}

Cada una de las tareas en este tablero puede ser asignada a una persona específica, también se pueden etiquetar dependiendo de la naturaleza de la tarea.

\subsubsection{Incremento} \label{increment}

\section{Metodología utilizada en el proyecto}