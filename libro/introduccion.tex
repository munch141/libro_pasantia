\chapter*{Introducción}

El proyecto presente, realizado en el marco de los Cursos de Cooperación Técnica y Desarrollo Social de la Universidad Simón Bolívar, consistió en el desarrollo de una aplicación web llamada \emph{eFuel} que sirve como apoyo al sistema de gestión de recursos de una empresas mayoristas de combustibles. La aplicación desarrollada permite a los usuarios realizar pedidos de combustible y monitorear el estado de dichos pedidos, de pagos y de facturas. Además, permite gestionar entidades del modelo de negocio como clientes, transportes y productos.

\section*{Antecedentes}
El proyecto se llevó a cabo en la empresa iKêls Consulting, la cual se especializa en el desarrollo de aplicaciones y sitios web para el sector corporativo utilizando Umbraco como la plataforma principal de manejo de  contenido (CMS). Umbraco es una herramienta de código abierto desarrollada en Dinamarca usando el lenguaje C\# y ASP.NET para sistemas Windows de Microsoft que cuenta con una gran aceptación a nivel mundial y una amplia comunidad de desarrolladores. Entre los activos de la empresa se encuentra una aplicación web (\emph{eFuel}) que permite a las estaciones de servicio de combustible la colocación de sus pedidos al mayorista. Dicha aplicación fue desarrollada hace más de 10 años usando tecnología Classic ASP y VBScript.

\section*{Planteamiento del problema}
Las empresas de distribución de combustible deben atender diariamente un volumen importante de pedidos a lo largo del territorio nacional. Este proceso involucra a múltiples actores, tiene un alto impacto en la sociedad y su coordinación es una labor tediosa y delicada que requiere personal especializado. Se requiere reducir los errores humanos, mejorar la eficiencia y aumentar la calidad y oportunidad de la información asociada al proceso.

\section*{Justificación e importancia}
Con los procesos manuales tradicionales, el proceso de pedidos de combustible consume mucho tiempo y esfuerzo, está sujeto a muchos errores debido a las múltiples variables involucradas y no tiene flexibilidad para responder con rapidez a los cambios propios de la demanda de los clientes (EE/SS). Debido al alto impacto que tiene en la dinámica diaria de las personas y negocios, la inversión en herramientas de apoyo producirá mejoras inmediatas en todos los aspectos del proceso.

\section*{Objetivo general}
Desarrollar una nueva versión de la aplicación web eFuel integrada con la plataforma de manejo de contenido (CMS) Umbraco para el ingreso y administración de pedidos de combustible (cisternas).

\section*{Objetivos específicos}
\begin{itemize}
    \item Crear el módulo de administración de clientes, productos, precios, vehículos, rutas y conductores.
    \begin{itemize}
        \item Creación, edición y eliminación de registros.
        \item Calcular costo de cisterna según volumen y precios de productos.
        \item Asignación de vehículos y tipos de producto a clientes.
        \item Listado de registros con mecanismo de búsqueda.
    \end{itemize}

    \item Crear el módulo de administración de pedidos.
    \begin{itemize}
        \item Crear, editar y eliminar pedidos mediante consola especializada.
        \item Opciones diferenciadas según perfil del usuario.
        \item Listado para consulta de pedidos con opción de exportar en formato MS Excel.
    \end{itemize}

    \item Crear el módulo de conciliación de pagos.
    \begin{itemize}
        \item Mecanismo para registro de pagos totales o parciales.
        \item Mecanismo de conciliación y asignación de pagos a pedidos.
    \end{itemize}

    \item Crear el módulo de registro de usuarios y perfiles.
    \begin{itemize}
        \item Registro y edición de usuarios con perfil de permisos.
        \item Asociar usuarios a clientes.
    \end{itemize}
\end{itemize}