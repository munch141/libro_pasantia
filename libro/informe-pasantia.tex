% Centro de Estadística y Matemática Aplicada
% Universidad Simón Bolívar
% Plantilla LaTeX para manuscritos (tesis y pasantías)
% pregrado y postgrado
%
% Andrés M. Sajo-Castelli
% Carlos Contreras
%
% 15 Abril 2015 -- primera versión pública
% ...
% 11 Mayo 2018 --- Se agrega bibliografía en castellano via babelbib
%
\documentclass[pregrado, table]{tesis-usb}
    
% paquetes
\usepackage[utf8]{inputenc}
\usepackage{ragged2e}
\usepackage{indentfirst}
\usepackage{longtable}
\usepackage{tabu}
\usepackage{float}
\usepackage{hhline}
\usepackage{makecell}
\usepackage{multirow}
\usepackage{array}
\usepackage{verbatim}
\usepackage{acronym}
\usepackage{amsmath}
\usepackage{amsfonts}
\usepackage{amssymb}
\usepackage{float}
\usepackage{pdfpages}
\usepackage{xcolor}
\definecolor{red}{HTML}{FF4C4C}


% estilo de las referencias
\usepackage[fixlanguage]{babelbib-and}\selectbiblanguage{spanish}
\usepackage{url}
\bibliographystyle{babplain-lf}

\usepackage{graphicx}
\graphicspath{ {../imgs/} }

% \usepackage{bookmark}

\autor{Ricardo Münch}
\autori{R. Münch}
\usbid{11-10684}
\titulo{Desarrollo de la versión 2 de la aplicación web eFuel}
\fecha{Septiembre~de~2018}
\agno{2018}
\fechadefensa{15~de~noviembre~de~2018}
\tutor{Tutor Académico: Prof. Soraya Carrasquel}
\usarcotutor
\cotutor{Tutor Industrial: Ing. José Cerqueiro}
\trabajo{Informe de Pasantía}
\coord{Ingeniería de la Computación}
\grado{Ingeniero en Computación}
\carrera{Ingeniería de la Computación}
\programa{Nombre del Programa}
\juradouno{Nombre y Apellido}
\juradodos{Nombre y Apellido}
\juradotres{Nombre y Apellido}

% Cambia comillas simple por comilla cerrada en ambiente verbatim
\makeatletter
\let \@sverbatim \@verbatim
\def \@verbatim {\@sverbatim \verbatimplus}
{\catcode`'=13 \gdef \verbatimplus{\catcode`'=13 \chardef '=13 }}
\makeatother

\begin{document}

    \frontmatter
    \maketitle
    \begin{resumen}
        El presente documento tiene por objetivo describir el proceso de desarrollo de la aplicación web eFuel durante el período Abril-Septiembre 2018 en la empresa iKêls Consulting, una empresa que se especializa en el desarrollo de aplicaciones y sitios web para el sector corporativo utilizando Umbraco como la plataforma principal de manejo de contenido (CMS). La aplicación desarrollada permite a las estaciones de servicio de combustible la colocación de sus pedidos al mayorista. Una primera versión de esta aplicación fue desarrollada hace más de 10 años utilizando tecnología obsoleta, por lo tanto, la empresa requiere de un proceso re-ingeniería completo para adaptar la solución a plataformas y modelos de gestión más modernos así como integrarla como un módulo o paquete del CMS Umbraco. El desarrollo se dividió en varias fases: inducción del pasante, análisis de requerimientos, diseño, implementación y documentación, la metodología para llevar a cabo estas fases se basa en las ideas de Scrum. En cuanto a las tecnologías utilizadas, la aplicación fue desarrollada sobre la plataforma Umbraco CMS que está, a su vez, construida sobre la plataforma ASP.NET de Microsoft, los lenguajes de programación utilizados fueron C\# para el back end y HTML, CSS y JavaScript para el front end, para la base de datos se usó el DBMS SQL Server de Microsoft. Como resultado del proyecto de pasantía se obtuvo una versión funcional de la solución y se cumplieron los objetivos planteados.
    \end{resumen}
    \tableofcontents
    \listoffigures

    \mainmatter
    \chapter*{Introducción}

El proyecto presente, realizado en el marco de los Cursos de Cooperación Técnica y Desarrollo Social de la Universidad Simón Bolívar, consistió en el desarrollo de una aplicación web llamada \emph{eFuel} que sirve como apoyo al sistema de gestión de recursos de una empresas mayoristas de combustibles. La aplicación desarrollada permite a los usuarios realizar pedidos de combustible y monitorear el estado de dichos pedidos, de pagos y de facturas. Además, permite gestionar entidades del modelo de negocio como clientes, transportes y productos.

\section*{Antecedentes}
El proyecto se llevó a cabo en la empresa iKêls Consulting, la cual se especializa en el desarrollo de aplicaciones y sitios web para el sector corporativo utilizando Umbraco como la plataforma principal de manejo de  contenido (CMS). Umbraco es una herramienta de código abierto desarrollada en Dinamarca usando el lenguaje C\# y ASP.NET para sistemas Windows de Microsoft que cuenta con una gran aceptación a nivel mundial y una amplia comunidad de desarrolladores. Entre los activos de la empresa se encuentra una aplicación web (\emph{eFuel}) que permite a las estaciones de servicio de combustible la colocación de sus pedidos al mayorista. Dicha aplicación fue desarrollada hace más de 10 años usando tecnología Classic ASP y VBScript.

\section*{Planteamiento del problema}
Las empresas de distribución de combustible deben atender diariamente un volumen importante de pedidos a lo largo del territorio nacional. Este proceso involucra a múltiples actores, tiene un alto impacto en la sociedad y su coordinación es una labor tediosa y delicada que requiere personal especializado. Se requiere reducir los errores humanos, mejorar la eficiencia y aumentar la calidad y oportunidad de la información asociada al proceso.

\section*{Justificación e importancia}
Con los procesos manuales tradicionales, el proceso de pedidos de combustible consume mucho tiempo y esfuerzo, está sujeto a muchos errores debido a las múltiples variables involucradas y no tiene flexibilidad para responder con rapidez a los cambios propios de la demanda de los clientes (EE/SS). Debido al alto impacto que tiene en la dinámica diaria de las personas y negocios, la inversión en herramientas de apoyo producirá mejoras inmediatas en todos los aspectos del proceso.

\section*{Objetivo general}
Desarrollar una nueva versión de la aplicación web eFuel integrada con la plataforma de manejo de contenido (CMS) Umbraco para el ingreso y administración de pedidos de combustible (cisternas).

\section*{Objetivos específicos}
\begin{itemize}
    \item Crear el módulo de administración de clientes, productos, precios, vehículos, rutas y conductores.
    \begin{itemize}
        \item Creación, edición y eliminación de registros.
        \item Calcular costo de cisterna según volumen y precios de productos.
        \item Asignación de vehículos y tipos de producto a clientes.
        \item Listado de registros con mecanismo de búsqueda.
    \end{itemize}

    \item Crear el módulo de administración de pedidos.
    \begin{itemize}
        \item Crear, editar y eliminar pedidos mediante consola especializada.
        \item Opciones diferenciadas según perfil del usuario.
        \item Listado para consulta de pedidos con opción de exportar en formato MS Excel.
    \end{itemize}

    \item Crear el módulo de conciliación de pagos.
    \begin{itemize}
        \item Mecanismo para registro de pagos totales o parciales.
        \item Mecanismo de conciliación y asignación de pagos a pedidos.
    \end{itemize}

    \item Crear el módulo de registro de usuarios y perfiles.
    \begin{itemize}
        \item Registro y edición de usuarios con perfil de permisos.
        \item Asociar usuarios a clientes.
    \end{itemize}
\end{itemize}
    \chapter{Entorno empresarial} \label{entornoEmpresarial}
En este capítulo se describe a la empresa en la cual se desarrolló el proyecto de pasantía. Comprende una breve reseña histórica, su misión y visión, la estructura organizacional y el área a la cual el pasante estuvo asignado.

\section{Antecedentes de la empresa} \label{antecedentesEmpresa}
IKêls Consulting \cite{ikelsAbout}  se creó en el año 2008 en Venezuela como una empresa dedicada al desarrollo de soluciones en el área de Sistemas de Información. Los fundadores contaban con una amplia trayectoria en los procesos y tecnología para la elaboración de documentación técnica avanzada (por ejemplo normas para construcción de plantas petroquímicas).

Para aprovechar la experiencia previa los productos y servicios se concentran en el área de aplicaciones web (por ejemplo, con \ac{CMS}) para el sector corporativo atendiendo a un selecto grupo de clientes con presencia local e internacional.

Actualmente, las actividades principales se concentran en:

\begin{itemize}
  \item Construcción de portales web en múltiples idiomas que pueden ser administrados por sus propios dueños. Esto incluye la programación de módulos especiales para integrar información desde y hacia sistemas externos, desplegar datos de manera amigable o generar notificaciones automáticas dependientes de actividades de los visitantes u otros eventos.
  \item Apoyo en la gestión de contenido de portales web.
  \item Consultoría y gestión para optimizar las variables asociadas al rendimiento y desempeño de las páginas web. Teniendo especial interés en el monitoreo de presencia en buscadores, evaluación del perfil de los visitantes y garantizar un nivel adecuado de usabilidad en diferentes dispositivos, etc.
  \item Desarrollo de productos personalizados que complementen las ventajas y facilidades de los dispositivos móviles en sincronización con mecanismos de soporte en servidores web.
  \item Desarrollo de soluciones especializadas para ofrecer bajo el modelo SaaS o Software as a Service.
\end{itemize}

\section{Misión}
Proveer productos y servicios en el área de sistemas de información que permitan una comunicación efectiva de nuestros clientes con su público y también sirva como plataforma de trabajo donde se aprovechen las innovaciones y ventajas de las tecnologías más modernas \cite{ikelsAbout}.

\section{Visión}
Desean ser proveedores confiables, que ofrecen un alto valor agregado en cada producto o servicio que prestan a sus clientes \cite{ikelsAbout}.

\section{Ubicación del pasante}
El proyecto de pasantía pertenece al grupo de desarrollo de aplicaciones y cuenta con la dirección de la Gerencia General y con el apoyo de los ingenieros líderes del grupo. En la Figura \ref{fig:organigrama} se muestra el organigrama de la empresa.

\begin{figure}[H]
\centering
    \includegraphics[width=.6\textwidth,height=.6\textheight,keepaspectratio]{organigrama.png}
    \caption{Organigrama de la empresa}
    \label{fig:organigrama}
\end{figure}
    \chapter{Marco Teórico}
En el presente capítulo se definen las bases teóricas sobre las cuáles se apoya el proyecto. Definiremos el tipo de sistema sobre el cual se desarrolló la aplicación, los patrones de diseño usados para la construcción de la aplicación y algunos conceptos útiles para entender algunos componentes del sistema desarrollado.

\section{Sistemas de Información}
    \subsection{CMS}
    Un CMS \cite{cmsBarker}, por sus siglas en inglés \textit{Content Management System} (Sistema de Gestión de Contenidos), es una aplicación de software que provee algún nivel de automatización a las tareas de manejo de contenido. Un CMS permite a los usuarios crear nuevo contenido, editar contenido existente, y hacer el contenido accesible al público.

    Un CMS consta, básicamente, de 2 partes:
    
    \paragraph*{Back end} \label{backEndCMS} es una interfaz para la edición de contenido, es la parte del sitio que permite crear y editar contenido.
    \paragraph*{Front end} es una interfaz para mostrar el contenido publicado, lo que el usuario final conoce como la \emph{página web} como tal.

    El uso de un CMS facilita las tareas de mantenimiento y de generación de contenido, especialmente para usuarios que no tienen preparación técnica especial. Los usuarios de la aplicación eFuel serán, en su mayoría, personal sin conocimientos especializados en el área de computación, ésta es una de las razones por las cuales resulta conveniente desarrollar el sistema sobre un CMS.

    \subsection{ERP} \label{erp}
    Un ERP \cite{internetEncyclopedia} o Sistema de Planificación de Recursos Empresariales (\emph{Enterprise Resource Planning por sus siglas en inglés}) es un sistema de gestión de los procesos integrales de un negocio. Le hace seguimiento a los recursos de la empresa (dinero, materia prima, capacidad de producción, etc) y al estado del negocio (pedidos, facturas, pagos de nómina, etc). Un ERP facilita el flujo de información entre los departamentos de una empresa. Normalmente se usan en empresas como es el caso de los distribuidoes de combustible. La aplicación eFuel interactúa con este tipo de sistemas y sirve de apoyo para llevar el control de pedidos, pagos y facturas.

\section{Aplicaciones Web}
    \subsection{API}
    Un API \cite{apiChristensson}, siglas en inglés para \textit{Application Programming Interface}, es un conjunto de comandos, funciones, protocolos y objetos que exponen los datos de una aplicación de software, es decir, establecen las reglas y los mecanismos a través de los cuales se puede tener acceso a ellos.

    Normalmente, aplicaciones externas disponen del API de una aplicación para obtener datos de esta última y usarlos para proveer algún servicio a sus usuarios. Para el proyecto presente se desarrolló un API para acceder a algunos de los datos de la misma.

    \subsection{Servicio Web} \label{webService}
    Un servicio web \cite{webServiceW3C} es un sistema de software diseñado para soportar interacción máquina-máquina a través de una red, proveen una vía estándar para la interoperabilidad entre distintas aplicaciones de software ejecutadas en distintas plataformas y ambientes. Típicamente, los sistemas externos tienen acceso al servicio web a través de un API.

    El proyecto presente incluye el desarrollo de un módulo de servicios web para el acceso a algunos datos de la aplicación, por ejemplo, datos para generar tablas informativas.

    \subsection{REST}
    Transferencia de Estado Representacional o REST \cite{restFielding}, por sus siglas en inglés, es un estilo de arquitectura para sistemas de hipermedia distribuidos (como la \textit{World Wide Web} o red informática mundial) que define una serie de restricciones que, cuando se aplican en conjunto, enfatizan la escalabilidad de interacciones entre componentes, la generalidad de las interfaces y el despliegue independiente de componentes.

    Las restricciones definidas para los sistemas REST son las siguientes:

    \begin{enumerate}
        \item \textbf{Separación Cliente-Servidor} el cliente y el servidor (ver Sección \ref{clientServer}) actúan independientemente, la interacción entre ellos ocurre solo a través de solicitudes, realizadas por el cliente, y respuestas, enviadas por el servidor como una reacción a una solicitud. El servidor solo envía información cuando ésta es solicitada por algún cliente.
        \item \textbf{Sin estado (\emph{stateless} en inglés)} el servidor no guarda información de ningún usuario que use los servicios del sistema. Cada solicitud individual contiene toda la información necesaria para ser ejecutada y enviar una respuesta, independientemente de todas las demás solicitudes que sean atendidas.
        \item \textbf{Permite el uso de memoria caché} la respuesta debe estar explícita o implícitamente etiquetada como \textit{cacheable} (se puede guardar en alguna memoria caché) o \textit{non-cacheable} (no se puede guardar en ninguna memoria caché). La ventaja del uso de memoria caché es que se pueden eliminar parcial o completamente algunas interacciones mejorando así el rendimiento del sistema.
        \item \textbf{Interfaz uniforme} cada solicitud al servidor debe tener los mismos 4 componentes: un identificador del recurso deseado, en el caso de aplicaciones web este identificador puede ser el URL; las respuesta del servidor debe incluir suficiente información para que el cliente pueda modificar el recurso; toda solicitud debe contener la información necesaria para que el servidor la pueda llevar a cabo y toda respuesta del servidor debe tener la información necesaria para que el cliente la entienda; uso de hipermedia como motor del estado de la aplicación, es decir, el servidor debe poder informar al cliente las formas en las que puede cambiar el estado de la aplicación a través de enlaces de hipermedia, una página web específica se puede considerar un estado de la aplicación y enlaces incluidos en esa página se consideran transiciones a otros estados de la aplicación.
        \item \textbf{Sistema por capas} entre el cliente que realiza una solicitud y el servidor que envía la respuesta final puede haber varios servidores, por ejemplo, puede haber un servidor que proporcione una capa de seguridad, otro una capa de memoria caché, y otros con otras funcionalidades. Estos servidores intermedios no deben afectar ni la solicitud ni la respuesta. Además, cada transacción (envío de la solicitud por el cliente y la subsiguiente respuesta) debe ser transparente para el cliente, es decir, el cliente no tiene conocimiento de las capas intermedias por las que pasan la solicitud y la respuesta.
    \end{enumerate}

    Se dice que un servicio web o el API de un servicio web es \textit{RESTful} cuando cumple con todas estas restricciones. El proyecto presente se desarrolló adhiriéndose a estas restricciones.

\section{Arquitectura}
A continuación se describen los modelos de arquitectura (física y de software) en los cuales se basa el funcionamiento y el desarrollo de la aplicación.

    \subsection{Modelo Cliente-Servidor} \label{clientServer}
    Arquitectura de redes de computadoras que divide el trabajo entre 2 entidades: un cliente y un servidor. El cliente le envía una solicitud al servidor y espera una respuesta. Por otra parte, el servidor recibe la solicitud, lleva a cabo el trabajo requerido y devuelve una respuesta al cliente \cite{redesTanenbaum}. El servidor mantiene una relación de uno-a-muchos con los clientes. Es importante destacar que los términos “cliente” y “servidor” pueden referirse tanto a máquinas como a programas o procesos.

    Esta arquitectura es ampliamente utilizada en aplicaciones web y el proyecto presente, siendo una aplicación web, usa este patrón, por lo que es importante tenerlo en mente para entender (y desarrollar) algunas de las funcionalidades de la aplicación.

    \subsection{MVC} \label{mvc}
    MVC \cite{mvcKrasner}, siglas para Modelo-Vista-Controlador, es un modelo de diseño de aplicaciones compuesto por 3 partes: el Modelo (datos), la Vista (presentación de los datos e interfaz con el usuario) y el Controlador (proceso que maneja la entrada de los usuarios y el acceso a los datos). Al separar la aplicación en estos 3 componentes se reduce la complejidad del diseño arquitectónico y se incrementa la reusabilidad, flexibilidad y mantenimiento del código. Adicionalmente, se pueden realizar cambios sobre un componente sin afectar a los demás, lo cual permite que cada componente tenga ciclos de desarrollo independientes.

    Actualmente, este patrón es ampliamente utilizado en el desarrollo de aplicaciones web ya que resulta muy natural acoplarlo con el modelo cliente-servidor que utiliza la web. El sistema eFuel fue desarrollado sobre una plataforma de desarrollo web que implementa el patrón MVC y, en consecuencia, el código tiene la estructura descrita por este patrón.
    \chapter{Marco Tecnológico}
En este capítulo se definen las tecnologías y herramientas utilizadas para el desarrollo del sistema.


\section{Lenguajes}
\subsection{C\#}
\subsection{HTML}
\subsection{CSS}
\subsection{JavaScript}

\section{Entorno de trabajo}
\subsection{.NET}
\subsection{ASP.NET}
\subsection{ASP.NET MVC}
\subsection{ASP.NET Web Api}
\subsection{ASP.NET Razor}
\subsection{Umbraco}

\section{}
\subsection{SQL Server}
\subsection{JSON}
\subsection{Ajax}
\subsection{JQuery}
\subsection{DataTables}


    \chapter{Marco Metodológico}
Para llevar a cabo los objetivos del proyecto se aplicó la metodología de desarrollo de software Proceso Unificado de Rational o RUP (por sus siglas en inglés) de IBM. A continuación se presentará brevemente la estructura de la metodología en cuestión, incluyendo sus fases y actividades asociadas.

\section{La metodología RUP}

\section{Fases de RUP}
\subsection{Iniciación}
\subsection{Elaboración}
\subsection{Construcción}
\subsection{Transición}

\section{Fases contempladas para el proyecto}
    \chapter{Desarrollo} \label{development}
En este capítulo se describe la aplicación web eFuel con sus módulos y funcionalidades, luego se narra cómo fue el desarrollo del proyecto. El desarrollo se divide en 3 fases: una fase de inducción donde el pasante se familiarizó con las herramientas y tecnologías necesarias para realizar el proyecto, una fase de desarrollo que consta de 6 Iteraciones de Scrum y una última fase de documentación.

A continuación la descripción de la aplicación seguido de las fases del desarrollo de la misma durante el proyecto de pasantía.

\section{La aplicación web eFuel}
\subsection{Versión original de eFuel}
Una primera versión de eFuel fue desarrollada por la empresa anteriormente y estaba  conformada por 4 módulos básicos:

\begin{itemize}
    \item \emph{Seguridad}: controlaba el acceso a laaplicación y sus datos.
    \item \emph{Pedidos}: inserción de pedidos de combustible.
    \item \emph{Pagos}: operaba como auxiliar contable del sistema de facturación interno de la organización (mayorista). 
    \item \emph{Datos}: ofrecía la posibilidad de activar y desactivar registros correspondientes a clientes, productos y otros parámetros no disponibles en el ERP de la organización. Adicionalmente poseía una bitácora que registraba las operaciones realizadas en el sistema.
\end{itemize}

Esta versión fue desarrollada hace más de 10 años utilizando Classic ASP, un motor de scripting para servidores cuya última versión fue desarrollada en el año 2000. Classic ASP fue reemplazada por ASP.NET en el año 2002, por esta razón no se pudo replicar la instalación de esta versión.

Para este proyecto se desea desarrollar una versión de la aplicación web eFuel basada en tecnologías modernas como ASP.NET y Umbraco. Se usaron documentos de la versión original para guiar el desarrollo de la nueva versión de eFuel, especialmente en el diseño de las entidades de la base datos y en las funcionalidades generales (módulos) del sistema.

\subsection{Nueva versión de eFuel}
La nueva versión de eFuel desarrollada para este proyecto es una aplicación web que permite a las estaciones de servicio de combustible la colocación de sus pedidos al mayorista. La aplicación permite consultar información del estado de los pedidos (si fueron despachados y si fueron pagados), asociar facturas a pedidos y asociar pagos a facturas, también permite consultar información sobre otras entidades del dominio de negocio como clientes (estaciones de servicio), transportes disponibles según la zona del cliente, productos disponibles y sus precios. Además, permite importar y exportar información para intercambiar con el sistema ERP del mayorista y, de esta manera, servir como sistema de apoyo para este.

\subsubsection{Módulos de funcionalidad}
El sistema se puede descomponer en 4 módulos de funcionalidad: Pedidos, Pagos, Administración y Seguridad. A continuación una breve descripción de las funcionalidades de cada módulo:

\begin{itemize}
    \item \emph{Pedidos}: módulo para la gestión de pedidos. Al crear un pedido se brinda la facilidad de seleccionar rápidamente la combinación de cliente-fecha-turno-transporte-productos. También permite mostrar un listado de los pedidos, filtrar la lista por cliente, transporte, rango de fecha o estado, y luego exportarla como un archivo de Excel. 
    \item \emph{Pagos}: módulo para la gestión de pagos y facturas. Con este módulo se pueden importar facturas desde un archivo de Excel generado por el sistema ERP del mayorista, estas se asocian inmediatamente con un pedido. También se pueden gestionar los pagos, asociar pagos a facturas y exportarlos a un archivo de Excel.
    \item \emph{Administración}: módulo para la gestión de clientes, transportes, productos, zonas y turnos. Esto incluye creación, edición, consulta, listado, importación y eliminación de estas entidades del sistema.
    \item \emph{Seguridad}: este módulo controla el acceso a la aplicación para garantizar que únicamente las personas autorizadas puedan consultar los datos y realizar las operaciones que les corresponden. Se pueden crear y editar miebros, así como modificar los permisos de cada uno.
\end{itemize}

En la figura \ref{fig:esquema_general_nuevo} se muestra el esquema general del sistema.

\begin{figure}[ht]
    \includegraphics[width=\textwidth]{esquema_general_nuevo.png}
    \caption{Esquema general de los módulos de funcionalidad}
    \label{fig:esquema_general_nuevo}
    \centering
\end{figure}

Uno de los objetivos de la aplicación es servir como sistema de apoyo a los ERP de los mayoristas. Los sistemas ERP llevan el control de todas las entidades del negocio, entre estas están las estaciones de servicio, los transportes para los pedidos, los pedidos, las facturas, etc. La aplicación eFuel interactúa con el sistema ERP y le sirve de apoyo a este manteniendo un registro separado de las entidades pertinentes a la inserción y el pago de pedidos de combustible. Esta interacción se da a través de importaciones y exportaciones de datos entre estos sistemas para que estén sincronizados. En la figura \ref{fig:domain_model} se describe esta interacción con el sistema ERP.

\section{Fases del desarrollo}
\subsection{Fase de Inducción}
Esta fase tuvo una duración de 1 semana, el pasante realizó varios tutoriales y utilizó varios recursos (guías de Umbraco, videos de ASP.NET, entre otros) proporcionados por la empresa para inducir los conocimientos técnicos necesarios para llevar a cabo el proyecto. Las herramientas investigadas fueron: Umbraco, ASP.NET, SQL Server y Visual Studio.

Aunque esta fase estuvo exclusivamente dedicada al aprendizaje hay que remarcar que, durante toda la pasantía, el pasante adquirió nuevas destrezas que fueron necesarias y facilitaron el desarrollo del proyecto. Una vez culminada esta fase se empezó el desarrollo de la aplicación.


\subsection{Fase de Desarrollo}
Esta fase duró 16 semanas continuas y se llevaron a cabo 6 Iteraciones donde se desarrolló la aplicación. A continuación una descripción del trabajo realizado en cada iteración.

\subsubsection{Análisis de requerimientos (1era Iteración)}
Se definió el alcance inicial del proyecto, la arquitectura y la estructura de la base de datos. Además, se creó el repositorio de Git, el proyecto de Visual Studio y el sitio de Umbraco, y por otro lado, se eligió una plantilla de HTML para el \emph{look and feel} de la aplicación.

\textbf{Actividades realizadas:}
\begin{itemize}
    \item Familiarización con la versión original de eFuel.
    \item Definir las reglas de negocio para el sistema.
    \item Definir los actores del sistema.
    \item Se listaron las funcionalidades básicas a desarrollar.
    \item Creación de la solución de Visual Studio.
    \item Creación del repositorio de Git y familiarización con las reglas del mismo.
    \item Se definieron las entidades de la base de datos del proyecto para poder implementar funcionalidades básicas del sistema.
    \item Se creó el sitio de Umbraco.
    \item Se evaluaron las plantillas para el front end de la aplicación y se eligió una.
\end{itemize}

\textbf{Duración:} 2 semanas.

\subsubsection{Módulo de Administración (2da Iteración)}
Se implementó la funcionalidad de manejo de las entidades principales de la aplicación: clientes, productos, transportes, zonas, turnos, pedidos, facturas y cobros. Para esto se implementó una interfaz en el back end de Umbraco usando Fluidity y se definieron los Doctypes y Datatypes de las entidades a ser almacenadas como contenido de Umbraco. También se empezó el desarrollo del front end de la aplicación. Al finalizar esta Iteración se tuvo una forma de manejar las entidades del sistema, ya sea a través del árbol de contenido de Umbraco o a través de la interfaz de Fluidity.

\textbf{Actividades realizadas:}
\begin{itemize}
    \item Se definieron los Doctypes y Datatypes para las entidades que serán manejadas a través de Umbraco.
    \item Se desarrolló la interfaz de Fluidity para insertar registros en la base de datos de las entidades que serán manejadas directamente con esta.
    \item Se empezó el desarrollo del front end de la aplicación, esto es, de los elementos en común que tendrán todas las páginas: navbar, título, etc.
    \item Se desarrollaron las vistas de listado de facturas, transportes y zonas en el front end.
\end{itemize}

\textbf{Duración:} 3 semanas.

\subsubsection{Módulo de Pedidos (3ra Iteración)}
Esta Iteración estuvo dedicada al desarrollo de la funcionalidad referente a los pedidos de combustible. Se desarrolló el listado de pedido con filtros, el formulario de creación de pedidos y la exportación de la lista de pedidos.

\textbf{Actividades realizadas:}
\begin{itemize}
    \item Desarrollo de la vista de listado de pedidos.
    \item Implementación de los filtros para la lista de pedidos.
    \item Implementación de la vista de detalles de un pedido.
    \item Mejoras al código y funcionalidad desarrolladas en las demás Iteraciones.
\end{itemize}

\textbf{Duración:} 2 semanas.

\subsubsection{Refactorización (4ta Iteración)}
En esta Iteración se determinó que la implementación de la funcionalidad desarrollada hasta ahora no estaba siguiendo la estructura de la solución de Visual Studio, es decir, los controladores no estaban bien organizados según el patrón MVC (como lo implementa ASP.NET) y que además había funcionalidad que debería estar implementada en \verb|EF_API| pero que estaba en \verb|EF_Core| y usando el tipo de controlador incorrecto. Esto se debió a un a falta de comunicación entre el dueño del producto y el pasante. Se realizaron las correcciones necesarias, esto tomó tiempo que no pudo ser invertido en el desarrollo de unos de los módulos planteados en el plan de trabajo original (Módulo de conciliación de pagos), en consecuencia, los casos de uso de este módulo quedaron parcialmente implementados.

\textbf{Actividades realizadas:}
\begin{itemize}
    \item Reorganización de los controladores de la aplicación.
    \item Refactorización del código en gran parte de la funcionalidad.
    \item Mejoras en la calidad del código.
\end{itemize}

\textbf{Duración:} 3 semanas.

\subsubsection{Continuación con el Módulo de Pedidos (5ta Iteración)}
En esta Iteración se continuó el desarrollo de la funcionalidad referente a los pedidos del sistema y se mejoraron algunos aspectos implementados en las otras Iteraciones.

\textbf{Actividades realizadas:}
\begin{itemize}
    \item Implementación del formulario de crear pedido.
    \item Implementación de la exportación de la lista de pedidos.
    \item Mejoras generales en el código de la aplicación.
\end{itemize}

\textbf{Duración:} 3 semanas.

\subsubsection{Módulo de Manejo de Usuarios e importación (6ta Iteración)}
Esta fue la última Iteración de desarrollo, en ella se desarrollo el módulo de manejo de usuarios y se implementó la funcionalidad de importación de algunas entidades. También se mejoraron todos los aspectos que fueron posibles del código.

\textbf{Actividades realizadas:}
\begin{itemize}
    \item Se definieron los miembros (Members de Umbraco) y los permisos por actor (customer y staff).
    \item Implementación de autenticación, inicio y cierre de sesión para el front end de la aplicación.
    \item Implementación de importación de Zonas y Transportes.
    \item Elaboración de la guía de instalación de la aplicación (ver anexo \ref{installationGuide}).
    \item Mejoras del código y corrección de bugs.
\end{itemize}

\textbf{Duración:} 3 semanas.

\subsection{Fase de Documentación} \label{documentation}
Esta fase duró el resto de las 20 semanas totales de la pasantía. En ella se documentó el estado final de la aplicación, se elaboró un Documento de Arquitectura de Software donde se detallan los componentes y casos de uso desarrollados de la aplicación y, además, se elaboró una guía de instalación de eFuel (ver apéndice \ref{installationGuide}). En total esto llevo 3 semanas de la pasantía y algunas extra después de culminada.

\subsubsection{Arquitectura del sistema}
A continuación se muestran el Modelo de Dominio y el Diagrama de Componentes del sistema que ayudan a describir la arquitectura del mismo y que fueron parte del trabajo realizado durante la fase de documentación (ver sección \ref{documentation}). El Modelo de Dominio ayuda a entender la relación de eFuel con el sistema ERP del mayorista. Por otro lado, el Diagrama de Componentes refleja como fue utilizado el patrón MVC en la arquitectura del sistema. Para una decripción detallada de la arquitectura del sistema referirse al anexo \ref{das}, donde se encuentra el Documento de Arquitectura de Software de eFuel.

\begin{figure}[H]
    \includegraphics[width=\textwidth]{domain_model.png}
    \caption{Modelo de dominio de eFuel}
    \label{fig:domain_model}
    \centering
\end{figure}

\begin{figure}[H]
    \includegraphics[width=\textwidth]{component_diagram.png}
    \caption{Diagrama de Componentes de eFuel}
    \label{fig:component_diagram}
    \centering
\end{figure}
    \chapter{Resultados} \label{results}
En este capítulo se presenta el estado en el que quedó la aplicación eFuel al finalizar el desarrollo. Se completó el 92\% de los casos de uso que tiene el sistema actualmente. No se realizaron pruebas de ningún tipo salvo por demostraciones del funcionamiento en una instalación local, sin embargo, el dueño del producto quedó satisfecho con el trabajo realizado y se espera que el desarrollo continue en el futuro. A continuación se presenta el diagrama de casos de uso y un resumen de la funcionalidad que posee el sistema actualmente. Además, se presentan 2 diagramas para ayudar a describir la estrctura del sistema: modelo de dominio y el diagrama de componentes. Para información detallada del funcionamiento e implementación del sistema referirse al Documento de Arquitectura de Software (anexo \ref{das}).

\section{Casos de Uso} \label{useCases}
En los siguientes diagramas se muestran en color verde los casos de uso que fueron implementados.

\begin{figure}[H]
    \includegraphics[width=\textwidth]{cu_admin_implementados.png}
    \caption{Casos de Uso back end (\emph{admin})}
    \label{fig:cu_admin_implementados}
    \centering
\end{figure}

\begin{figure}[H]
    \includegraphics[width=\textwidth]{cu_customer_staff_implementados.png}
    \caption{Casos de Uso front end (\emph{customer} y \emph{staff})}
    \label{fig:cu_customer_staff_implementados}
    \centering
\end{figure}

\section{Estado actual de la aplicación}
Como resultado del desarrollo del proyecto se obtuvo una aplicación web desarrollada nuevamente que funciona como un agregado a un sitio de Umbraco y que contiene, casi en su totalidad, las funcionalidades requeridas por el dueño del producto. No se ha desplegado la aplicación en un ambiente de producción, solo fue probada en varias instalaciones locales, a pesar de esto el dueño del producto quedó satisfecho con los resultados obtenidos. Sigue un resumen de la funcionalidad que tiene la aplicación, dividido en partes: funcionalidad en el front end de eFuel y funcionalidad en el back end de Umbraco.

\subsection{Funcionalidad en el front end}
\begin{itemize}
    \item Creación de pedidos según la disponibilidad de transportes, fechas y turnos para un cliente.
    \item Listado de pedidos con filtros por cliente, rango de fecha, transporte y estado.
    \item Exportación de un listado de pedidos con filtros aplicados.
    \item Visualización de la lista de clientes, transportes, zonas y facturas.
    \item Importación de lista de transportes y zonas.
\end{itemize}

\subsection{Funcionalidad en el back end de Umbraco}
\begin{itemize}
    \item Listado de clientes, transportes, zonas, turnos y productos.
    \item Creación, edición, vista de detalles y eliminación de clientes, transportes, zonas, turnos y productos.
    \item Listado de pedidos, facturas y cobros.
    \item Creación, edición, vista de detalles y eliminación de pedidos, facturas y cobros.
    \item 
\end{itemize}

    \chapter*{Conclusiones y Recomendaciones}
Este proyecto de pasantía consistió en desarrollar una segunda versión de una aplicación web llamada eFuel. La aplicación permite, a estaciones de servicio de combustible, realizar pedidos de combustible a las plantas de producción. También, permite realizar actividades de administración de pagos y facturas para estos pedidos. La aplicación fue desarrollada en la empresa iKêls Consulting sobre la plataforma de manejo de contenido Umbraco. Se siguió la metodología Scrum durante el transcurso de la pasantía, en total se llevaron a cabo 6 Iteraciones a lo largo de 20 semanas para el desarrollo de la aplicación.

Durante el desarrollo hubieron algunos retrasos y refactorizaciones en el código (de hecho no se pudo completar la funcionalidad del Módulo de Pagos), esto se puede atribuir a varios factores. Se hicieron unas estimaciones de tiempo para el desarrollo de cada módulo, estas resultaron ser muy cortas. Por otro lado, se llevaron a cabo revisiones del código que hicieron aparente la necesidad de refactorizarlo. Con esta refactorización, se buscaba seguir los principios de MVC sobre los cuáles está diseñado el framework ASP.NET MVC y, de esta manera, producir código mantenible y de buena calidad. 

Los requerimientos del producto fueron cambiando a lo largo del desarrollo, esto se puede atribuir, en parte, a que no se elaboró documentación adecuada al inicio del desarrollo. Sabiendo esto, hay que resaltar la importancia de generar documentos al inicio del desarrollo que ayuden a plasmar una visión general del sistema para que todos los actores tengan claro qué hace el sistema y cómo se espera que lo haga, de esta manera el desarrollo será más fluido. 

Al enfrentarse con nuevas tecnologías y conceptos, es importante saber consultar tanto fuentes de información (documentación, libros, etc) como al equipo de trabajo, ya que muchas veces otra persona del equipo ha resuelto problemas similares o tiene experiencia con las tecnologías utilizadas y puede ayudar a avanzar con el desarrollo. Hay que saber cómo y dónde buscar información.

Como recomendaciones se tiene lo siguiente:
\begin{itemize}
    \item Tener documentación adecuada y actualizada del sistema desde el inicio, esto facilita el desarrollo y provee claridad para todos involucrados con el producto.
    \item Realizar pruebas de todo tipo sobre la aplicación antes de continuar con el desarrollo, ya que no se realizaron pruebas durante la pasantía.
    \item Elaborar una guía de usuario.
    \item Proveer más opciones de gestión de datos en el front end de la aplicación para que el usuario no deba alternar entre la interfaz de Umbraco y el front end de eFuel.
    \item Tomar pasos para que la aplicación sea más fácil de instalar en un sitio de Umbraco, apuntar a hacer la aplicación un paquete de Umbraco.
\end{itemize}

El pasante obtuvo muchas enseñanzas tanto a nivel tecnológico como a nivel personal, sobre todo con respecto a lo que implica el proceso de desarrollo de software desde la concepción hasta la entrega del producto. La documentación y la planificación son integrales para poder crear un producto de buena calidad y en tiempos deseables, se debe tener siempre en mente que muchos componentes del software cambian a lo largo del desarrollo, también es importante saber cuándo se debe pedir ayuda a otras personas del equipo para poder agilizar el desarrollo de la mejor manera posible.

El proyecto de pasantía fue una oportunidad para poner en práctica muchos conceptos y teorías aprendidas en la carrera, para reforzar y complementarlos, para entender por qué son importantes; pero también que la teoría y la práctica pueden ser muy diferentes y es el trabajo del ingeniero reconciliar las 2 para mejorar siempre los procesos de desarrollo y sus resultados.
    \nocite{*}
    \bibliography{referencias}
    \appendix
    \chapter{VISTAS DE EFUEL} \label{vistas}
\section{Front end}
\begin{figure}[H]
    \includegraphics[width=\textwidth]{vistas/frontend/login.png}
    \caption{Inicio de Sesión}
    \label{fig:frontend:login}
    \centering
\end{figure}

\begin{figure}[H]
    \includegraphics[width=\textwidth]{vistas/frontend/orders_list.png}
    \caption{Lista de pedidos}
    \label{fig:frontend:orders_list}
    \centering
\end{figure}

\begin{figure}[H]
    \includegraphics[width=\textwidth]{vistas/frontend/orders_create.png}
    \caption{Crear un pedido}
    \label{fig:frontend:orders_create}
    \centering
\end{figure}

\begin{figure}[H]
    \includegraphics[width=\textwidth]{vistas/frontend/orders_create_preview.png}
    \caption{Crear un pedido - Preview}
    \label{fig:frontend:orders_create_preview}
    \centering
\end{figure}

\begin{figure}[H]
    \includegraphics[width=\textwidth]{vistas/frontend/invoices_list.png}
    \caption{Lista de facturas}
    \label{fig:frontend:invoices_list}
    \centering
\end{figure}

\begin{figure}[H]
    \includegraphics[width=\textwidth]{vistas/frontend/customers_list.png}
    \caption{Lista de clientes}
    \label{fig:frontend:customers_list}
    \centering
\end{figure}

\begin{figure}[H]
    \includegraphics[width=\textwidth]{vistas/frontend/customers_detail.png}
    \caption{Detalles de cliente}
    \label{fig:frontend:customers_detail}
    \centering
\end{figure}

\begin{figure}[H]
    \includegraphics[width=\textwidth]{vistas/frontend/transports_list.png}
    \caption{Dashboard de transportes}
    \label{fig:frontend:transports_list}
    \centering
\end{figure}

\begin{figure}[H]
    \includegraphics[width=\textwidth]{vistas/frontend/zones_list.png}
    \caption{Dashboard de zonas}
    \label{fig:frontend:zones_list}
    \centering
\end{figure}

\section{Back end (Umbraco)}
\begin{figure}[H]
    \includegraphics[width=\textwidth]{vistas/backend/login.png}
    \caption{Inicio de sesión (Umbraco)}
    \label{fig:backend:login}
    \centering
\end{figure}

\begin{figure}[H]
    \includegraphics[width=\textwidth]{vistas/backend/list.png}
    \caption{Lista de clientes}
    \label{fig:backend:list}
    \centering
\end{figure}

\begin{figure}[H]
    \includegraphics[width=\textwidth]{vistas/backend/details.png}
    \caption{Detalles de cliente}
    \label{fig:backend:details}
    \centering
\end{figure}

\begin{figure}[H]
    \includegraphics[width=\textwidth]{vistas/backend/edit.png}
    \caption{Editar cliente}
    \label{fig:backend:edit}
    \centering
\end{figure}

\begin{figure}[H]
    \includegraphics[width=\textwidth]{vistas/backend/members.png}
    \caption{Administración de miembros}
    \label{fig:backend:members}
    \centering
\end{figure}

\begin{figure}[H]
    \includegraphics[width=\textwidth]{vistas/backend/flu_list.png}
    \caption{Lista de pedidos (Fluidity)}
    \label{fig:backend:flu_list}
    \centering
\end{figure}

\begin{figure}[H]
    \includegraphics[width=\textwidth]{vistas/backend/flu_detail.png}
    \caption{Detalles de pedido (Fluidity)}
    \label{fig:backend:flu_detail}
    \centering
\end{figure}
    \chapter{DOCUMENTO DE ARQUITECTURA DE SOFTWARE} \label{das}
\includepdf[pages=-]{../anexos/das/das.pdf}

\end{document}
