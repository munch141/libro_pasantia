\chapter{Resultados} \label{results}
Se completó el 84\% de los casos de uso descritos en Documento de Arquitectura del Software \ref{das}. No se completaron en su totalidad porque hubo un retraso importante durante el desarrollo debido a una refactorización del código que se realizó a mediados del desarrollo, otra razón importante es que las estimaciones del tiempo requerido para implementar las funcionalidades fueron muy cortas. Los casos de uso que no se implementaron pertenecen, en su mayoría, al Módulo de Pagos.

No se realizaron pruebas unitarias ni de integración debido a los atrasos con el desarrollo. Se determinó que las pruebas no eran prioritarias; era preferible implementar toda la funcionalidad posible. Ésta fue una decisión tomada en conjunto con el dueño del producto. Sin embargo, sí se realizaron demostraciones del funcionamiento en varias máquinas instalando el sitio localmente.

\section{Estado actual de la aplicación}
Como resultado del desarrollo del proyecto, se obtuvo una versión funcional de la aplicación web eFuel que abarca el 84\% de los casos de uso estipulados. Este alcance fue acordado por la empresa y el pasante; la empresa quedó satisfecha con el trabajo realizado. En la Tabla \ref{tab:casosDeUso} se muestran los casos de uso del sistema y se resaltan los que \emph{no} fueron implementados en color rojo.

\newcounter{usecasesrownumbers}
\newcommand\usecasenumber{\stepcounter{usecasesrownumbers}\arabic{usecasesrownumbers}}

\vspace{0.1cm}
\begin{longtable}{ | l | l | c | }
    \caption{Resumen de casos de uso eFuel (en rojo los no implementados)}
    \label{tab:casosDeUso} \\

    \hline
    \rowcolor{gray!30}
    \multicolumn{1}{|c|}{ID del Caso de Uso} &
    \multicolumn{1}{|c|}{Caso de Uso} &
    \multicolumn{1}{|c|}{Actor} \\
    \hhline{===}
    \endhead

    CU-\usecasenumber & Iniciar sesión (Umbraco) & Admin \\ \hline
    CU-\usecasenumber & Consultar lista de miembros & Admin \\ \hline
    CU-\usecasenumber & Gestionar miembro (CRUD) & Admin \\ \hline
    CU-\usecasenumber & Asignar cliente/s a miembro & Admin \\ \hline
    CU-\usecasenumber & Remover cliente/s de miembro & Admin \\ \hline
    CU-\usecasenumber & Cambiar permisos de miembro & Admin \\ \hline

    CU-\usecasenumber & Gestionar contenido & Admin \\ \hline
    CU-\usecasenumber & Consultar lista de clientes & Admin \\ \hline
    CU-\usecasenumber & Gestionar cliente (CRUD) & Admin \\ \hline
    CU-\usecasenumber & Consultar lista de productos & Admin \\ \hline
    CU-\usecasenumber & Gestionar producto (CRUD) & Admin \\ \hline
    CU-\usecasenumber & Consultar lista de transportes & Admin \\ \hline
    CU-\usecasenumber & Gestionar transportes (CRUD) & Admin \\ \hline
    CU-\usecasenumber & Consultar lista de zonas & Admin \\ \hline
    CU-\usecasenumber & Gestionar zonas (CRUD) & Admin \\ \hline

    CU-\usecasenumber & Gestionar transacciones & Admin \\ \hline
    CU-\usecasenumber & Consultar lista de registros & Admin \\ \hline
    CU-\usecasenumber & Gestionar registro (CRUD) & Admin \\ \hline
    CU-\usecasenumber & Consultar lista de pedidos & Admin \\ \hline
    CU-\usecasenumber & Gestionar pedido (CRUD) & Admin \\ \hline
    CU-\usecasenumber & Consultar lista de detalles de pedidos & Admin \\ \hline
    CU-\usecasenumber & Gestionar detalle de pedido (CRUD) & Admin \\ \hline
    CU-\usecasenumber & Consultar lista de facturas & Admin \\ \hline
    CU-\usecasenumber & Gestionar factura (CRUD) & Admin \\ \hline
    
    \rowcolor{red}
    CU-\usecasenumber & Registrar pago & Customer, Staff \\ \hline
    \rowcolor{red}
    CU-\usecasenumber & Asignar pago a pedido & Customer, Staff \\ \hline
    \rowcolor{red}
    CU-\usecasenumber & Consultar lista de pagos & Customer, Staff \\ \hline
    \rowcolor{red}
    CU-\usecasenumber & Consultar pagos pendientes & Customer, Staff \\ \hline
    \rowcolor{red}
    CU-\usecasenumber & Consultar pagos realizados & Customer, Staff \\ \hline
    \rowcolor{red}
    CU-\usecasenumber & Registrar pago & Customer, Staff \\ \hline
    \rowcolor{red}
    CU-\usecasenumber & Importar lista de facturas & Staff \\ \hline
    
    CU-\usecasenumber & Consultar lista de cobros & Admin \\ \hline
    CU-\usecasenumber & Gestionar cobro (CRUD) & Admin \\ \hline
    CU-\usecasenumber & Consultar lista de detalles de cobros & Admin \\ \hline
    CU-\usecasenumber & Gestionar detalle de cobro (CRUD) & Admin \\ \hline

    CU-\usecasenumber & Iniciar sesión (eFuel) & Customer, Staff \\ \hline
    CU-\usecasenumber & Consultar lista de pedidos & Customer, Staff \\ \hline
    CU-\usecasenumber & Consultar pedido & Customer, Staff \\ \hline
    CU-\usecasenumber & Filtrar lista de pedidos & Customer, Staff \\ \hline
    CU-\usecasenumber & Exportar lista de pedidos & Customer, Staff \\ \hline
    CU-\usecasenumber & Crear pedido & Customer, Staff \\ \hline
    CU-\usecasenumber & Seleccionar cliente & Customer, Staff \\ \hline
    CU-\usecasenumber & Seleccionar fecha & Customer, Staff \\ \hline
    CU-\usecasenumber & Seleccionar turno  & Customer, Staff \\ \hline
    CU-\usecasenumber & Seleccionar transporte  & Customer, Staff \\ \hline
    CU-\usecasenumber & Seleccionar productos & Customer, Staff \\ \hline

    CU-\usecasenumber & Consultar lista de facturas & Customer, Staff \\ \hline

    CU-\usecasenumber & Consultar lista de clientes & Customer, Staff \\ \hline
    CU-\usecasenumber & Consultar cliente & Customer, Staff \\ \hline
    CU-\usecasenumber & Consultar pedidos de cliente & Customer, Staff \\ \hline
    \rowcolor{red}
    CU-\usecasenumber & Importar lista de clientes & Staff \\ \hline

    CU-\usecasenumber & Consultar lista de transportes & Customer, Staff \\ \hline

    CU-\usecasenumber & Importar lista de transportes & Staff \\ \hline

    CU-\usecasenumber & Consultar lista de zonas & Customer, Staff \\ \hline

    CU-\usecasenumber & Importar lista de zonas & Staff \\ \hline

    \rowcolor{red}
    CU-\usecasenumber & Importar lista de despachos & Staff \\ \hline
\end{longtable}

La nueva aplicación desarrollada funciona como un agregado a sitios de Umbraco. No se ha desplegado en un ambiente de producción, solo fue probada en varias instalaciones locales, a pesar de esto el dueño del producto quedó satisfecho con los resultados obtenidos.

En resumen, la funcionalidad desarrollada en el front end y el back office de Umbraco, para cada módulo, es la siguiente:

\begin{itemize}
    \item Front end:
    \begin{itemize}
        \item \emph{Seguridad}: inicio de sesión con credenciales.
        
        \item \emph{Pedidos}: creación de pedidos según la disponibilidad de transportes, fechas y turnos para un cliente; listado de pedidos con filtros por cliente, rango de fecha, transporte y estado; exportación de un listado de pedidos con filtros aplicados.
    
        \item \emph{Pagos}: listado de facturas.
        
        \item \emph{Administración}: listado de clientes, transportes y zonas; vista de detalles de un cliente; importación de lista de transportes y zonas.
    \end{itemize}

    \item Back office de Umbraco:
    \begin{itemize}
        \item \emph{Seguridad}: inicio de sesión con credenciales; administración (creación, edición, detalles, eliminación y permisos) de miembros de eFuel.
        
        \item \emph{Pedidos}: listado de pedidos; administración (creación, edición, vista de detalles y eliminación) de pedidos.
        
        \item \emph{Pagos}: listado de facturas y cobros; administración (creación, edición, vista de detalles y eliminación) de facturas y cobros.
        
        \item \emph{Administración}: listado de clientes, transportes, zonas, turnos y productos; administración (creación, edición, vista de detalles y eliminación) de clientes, transportes, zonas, turnos y productos.
    \end{itemize}
\end{itemize}

En el Anexo \ref{vistas} se encuentran algunas capturas de pantalla de estas vistas y funcionalidades.