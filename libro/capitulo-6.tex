\chapter{Resultados} \label{results}
En este capítulo se presenta el estado en el que quedó la aplicación eFuel al finalizar el desarrollo.

Se completó el xx\% de los casos de uso que tiene el sistema actualmente. Esto se debió a varios factores. Primero...

No se realizaron pruebas unitarias ni de integración debido a los atrasos con el desarrollo de las funcionalidades principales. Se determinó que las pruebas no eran prioritarias, era preferible implementar la mayor cantidad de funcionalidades restantes, ésta fue una decisión tomada en conjunto con el tutor industrial. Sin embargo, si se realizaron demostraciones del funcionamiento en varias máquinas instalando el sitio localmente.

\section{Estado actual de la aplicación}
Como resultado del desarrollo del proyecto se obtuvo una versión funcional de la aplicación web eFuel que abarca el xx\% de los casos de uso estipulados. Este alcance fue acordado por la empresa y el pasante, la empresa quedó satisfecha con el trabajo realizado. En la tabla \ref{tab:casosDeUso} se muestranlos casos de uso del sistema y se resaltan los que fueron implementados.

\newcounter{magicrownumbers}
\newcommand\rownumber{\stepcounter{magicrownumbers}\arabic{magicrownumbers}}

\begin{longtable}{ | l | l | c | }
    \hline
    % \rowcolor{gray!30}
    \multicolumn{1}{|c|}{ID del Caso de Uso} &
    \multicolumn{1}{|c|}{Caso de Uso} &
    \multicolumn{1}{|c|}{Actor} \\
    \hhline{===}
    \endhead

    CU-\rownumber & Iniciar sesión (Umbraco) & Admin \\ \hline
    CU-\rownumber & Consultar lista de miembros & Admin \\ \hline
    CU-\rownumber & Gestionar miembro (CRUD) & Admin \\ \hline
    CU-\rownumber & Asignar cliente/s a miembro & Admin \\ \hline
    CU-\rownumber & Remover cliente/s de miembro & Admin \\ \hline
    CU-\rownumber & Cambiar permisos de miembro & Admin \\ \hline

    CU-\rownumber & Gestionar contenido & Admin \\ \hline
    CU-\rownumber & Consultar lista de clientes & Admin \\ \hline
    CU-\rownumber & Gestionar cliente (CRUD) & Admin \\ \hline
    CU-\rownumber & Consultar lista de productos & Admin \\ \hline
    CU-\rownumber & Gestionar producto (CRUD) & Admin \\ \hline
    CU-\rownumber & Consultar lista de transportes & Admin \\ \hline
    CU-\rownumber & Gestionar transportes (CRUD) & Admin \\ \hline
    CU-\rownumber & Consultar lista de zonas & Admin \\ \hline
    CU-\rownumber & Gestionar zonas (CRUD) & Admin \\ \hline

    CU-\rownumber & Gestionar transacciones & Admin \\ \hline
    CU-\rownumber & Consultar lista de registros & Admin \\ \hline
    CU-\rownumber & Gestionar registro (CRUD) & Admin \\ \hline
    CU-\rownumber & Consultar lista de pedidos & Admin \\ \hline
    CU-\rownumber & Gestionar pedido (CRUD) & Admin \\ \hline
    CU-\rownumber & Consultar lista de detalles de pedidos & Admin \\ \hline
    CU-\rownumber & Gestionar detalle de pedido (CRUD) & Admin \\ \hline
    CU-\rownumber & Consultar lista de facturas & Admin \\ \hline
    CU-\rownumber & Gestionar factura (CRUD) & Admin \\ \hline
    CU-\rownumber & Importar lista de facturas & Staff \\ \hline
    CU-\rownumber & Reportar pago & Customer, Staff \\ \hline
    CU-\rownumber & Consultar lista de cobros & Admin \\ \hline
    CU-\rownumber & Gestionar cobro (CRUD) & Admin \\ \hline
    CU-\rownumber & Consultar lista de detalles de cobros & Admin \\ \hline
    CU-\rownumber & Gestionar detalle de cobro (CRUD) & Admin \\ \hline

    CU-\rownumber & Iniciar sesión (eFuel) & Customer, Staff \\ \hline
    CU-\rownumber & Consultar lista de pedidos & Customer, Staff \\ \hline
    CU-\rownumber & Consultar pedido & Customer, Staff \\ \hline
    CU-\rownumber & Filtrar lista de pedidos & Customer, Staff \\ \hline
    CU-\rownumber & Exportar lista de pedidos & Customer, Staff \\ \hline
    CU-\rownumber & Crear pedido & Customer, Staff \\ \hline
    CU-\rownumber & Seleccionar cliente & Customer, Staff \\ \hline
    CU-\rownumber & Seleccionar fecha & Customer, Staff \\ \hline
    CU-\rownumber & Seleccionar turno  & Customer, Staff \\ \hline
    CU-\rownumber & Seleccionar transporte  & Customer, Staff \\ \hline
    CU-\rownumber & Seleccionar productos & Customer, Staff \\ \hline

    CU-\rownumber & Consultar lista de facturas & Customer, Staff \\ \hline

    CU-\rownumber & Consultar lista de clientes & Customer, Staff \\ \hline
    CU-\rownumber & Consultar cliente & Customer, Staff \\ \hline
    CU-\rownumber & Consultar pedidos de cliente & Customer, Staff \\ \hline
    CU-\rownumber & Importar lista de clientes & Staff \\
    \hline

    CU-\rownumber & Consultar lista de transportes & Customer, Staff \\ \hline

    CU-\rownumber & Importar lista de transportes & Staff \\ \hline

    CU-\rownumber & Consultar lista de zonas & Customer, Staff \\ \hline

    CU-\rownumber & Importar lista de zonas & Staff \\
    \hline

    CU-\rownumber & Importar lista de despachos & Staff \\
    \hline

    CU-\rownumber & Reportar pagos & Customer, Staff \\
    \hline

    \caption{Resumen de casos de uso eFuel (en verde los implementados)}
    \label{tab:casosDeUso}
\end{longtable}

La aplicación desarrollada nuevamente que funciona como un agregado a un sitio de Umbraco. No se ha desplegado en un ambiente de producción, solo fue probada en varias instalaciones locales, a pesar de esto el dueño del producto quedó satisfecho con los resultados obtenidos. Sigue un resumen de la funcionalidad por módulos desarrollada, dividido en partes: funcionalidad en el front end de eFuel y funcionalidad en el back office de Umbraco.

\subsection{Funcionalidad en el front end}
\begin{itemize}
    \item \emph{Seguridad}: inicio de sesión con credenciales.
    
    \item \emph{Pedidos}: creación de pedidos según la disponibilidad de transportes, fechas y turnos para un cliente; listado de pedidos con filtros por cliente, rango de fecha, transporte y estado; exportación de un listado de pedidos con filtros aplicados.

    \item \emph{Pagos}: listado de facturas.
    
    \item \emph{Administración}: listado de clientes, transportes y zonas; vista de detalles de un cliente; importación de lista de transportes y zonas.
\end{itemize}

\subsection{Funcionalidad en el back office de Umbraco}
\begin{itemize}
    \item \emph{Seguridad}: inicio de sesión con credenciales; administración (creación, edición, detalles, eliminación y permisos) de miembros de eFuel.
    
    \item \emph{Pedidos}: listado de pedidos; administración (creación, edición, vista de detalles y eliminación) de pedidos.
    
    \item \emph{Pagos}: listado de facturas y cobros; administración (creación, edición, vista de detalles y eliminación) de facturas y cobros.
    
    \item \emph{Administración}: listado de clientes, transportes, zonas, turnos y productos; administración (creación, edición, vista de detalles y eliminación) de clientes, transportes, zonas, turnos y productos.
\end{itemize}

En el anexo \ref{vistas} se encuentran las capturas de pantalla de las funcionalidades enumeradas en esta sección.