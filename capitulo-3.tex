\chapter{Marco Tecnológico}
En este capítulo se definen las tecnologías y herramientas utilizadas para el desarrollo del sistema.


\section{Entorno de trabajo}
\subsection{ASP.NET}
ASP.NET es un modelo de desarrollo Web unificado que incluye los servicios necesarios para construir aplicaciones Web empresariales con un mínimo de codificación. ASP.NET es parte de .NET Framework y al codificar las aplicaciones en ASP.NET se tiene acceso a las clases de .NET Framework. \cite{asp.netMicrosoft}

\subsection{ASP.NET MVC}
ASP.NET MVC es un marco de trabajo (o \textit{framework} en inglés) para construir aplicaciones web escalables, basadas en estándares y usando patrones de diseño bien establecidos (el patrón MVC) y el poder de ASP.NET y .NET Framework. \cite{asp.netMVCMicrosoft}

\emph{Poner donde se usa}

\subsection{ASP.NET Web Api}
ASP.NET Web API es un marco de trabajo que facilita la construcción de servicios HTTP que llegan a una amplia gama de clientes, incluyendo navegadores y dispositivos móbiles. ASP.NET Web API es una plataforma ideal para construir aplicaciones RESTful en .NET Framework. \cite{asp.netWebAPIMicrosoft}

\emph{Poner donde se usa}

\subsection{Umbraco}
No consigo una fuente en donde esté la definición de Umbraco explicicitamente.

\section{Lenguajes}
\subsection{C\#}
C\# es un lenguaje con seguridad de tipos y orientado a objetos que permite a desarrolladores construir una variedad de aplicaciones robustas y seguras que corren sobre .NET Framework. \cite{}

Este es el lenguaje que se usa para el código de la aplicación.

\subsection{HTML}
Definición de HTML.

\subsection{CSS}
Definición de CSS.

\subsection{JavaScript}
Definición de JavaScript.