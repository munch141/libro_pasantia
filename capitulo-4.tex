\chapter{Marco Metodológico}

En este capítulo se describirá de manera global el marco de desarrollo Scrum que fue utilizado como metodología para el desarrollo del sistema, así como las actividades y resultados de cada una de sus etapas. El capítulo empezará con la definición de Scrum, luego se describirán los papeles o roles de un equipo Scrum, seguido de los eventos que ocurren durante Scrum, y finalmente los artefactos necesarios para este proceso. \textit{Mejor intro para esta parte}

\section{Definición de Scrum}
Scrum es un marco de trabajo dentro del cual la gente puede abordar un problema adaptativo complejo, mientras se entregan productos del mayor valor posible. Es un marco de trabajo para procesos que ha sido usado para gestionar el trabajo en productos complejos desde los inicios de la década de los 90. Scrum no es un proceso, ni una técnica o método definitivo

\section{Un equipo de Scrum}

\section{Eventos de Scrum}

\section{Artefactos de Scrum}