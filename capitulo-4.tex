\chapter{Marco Metodológico}

En este capítulo se describirá de manera global el marco de desarrollo Scrum que fue utilizado como metodología para el desarrollo del sistema, así como las actividades y resultados de cada una de sus etapas. El capítulo empezará con la definición de Scrum, luego se describirán los papeles o roles de un equipo Scrum, seguido de los eventos que ocurren durante Scrum, y finalmente los artefactos necesarios para este proceso. \textit{Mejor intro para esta parte}

\section{Definición de Scrum}
Scrum es un marco de trabajo dentro del cual la gente puede abordar un problema adaptativo complejo, mientras se entregan productos del mayor valor posible. Es un marco de trabajo para procesos que ha sido usado para gestionar el trabajo en productos complejos desde los inicios de la década de los 90. Scrum no es un proceso, ni una técnica o método definitivo. Más bien, es un marco de trabajo dentro del cual se pueden emplear varias técnicas y procesos. El marco de Scrum consiste en Equipos Scrum y sus roles, eventos, artefactos y reglas asociados. \cite{scrumSchwaber}

\section{Artefactos de Scrum}
\subsection{Lista de objetivos del producto}
La lista de objetivos del producto es una lista ordenada de todas las tareas que deben ser completadas para entregar el producto. Esta lista nunca está completa, la primera versión muestra los primeros requerimientos conocidos y mejor entendidos. \cite{scrumSchwaber} La lista de objetivos del producto evoluciona junto con el producto para actualizar los requerimientos dependiendo de las exigencias del cliente y del mercado.

\subsection{Lista de tareas de la iteración}
La loita de tareas de la iteración es el conjunto de tareas de la lista de objetivos del producto seleccionadas para completar en un Sprint, puede verse como una predicción de lo que estará completado al llevarse a cabo es Sprint. \cite{scrumSchwaber} Esta lista va a ir cambiando a medida que se identifiquen tareas y trabajo que debe ser realizado para completar los items durante el desarrollo del Sprint.

\subsection{Tablero de tareas}
El tablero de tareas es una tabla utilizada para gestionar el estado de los objetivos de la lista de objetivos del producto, contiene 4 columnas:

\begin{itemize}
    \item \textbf{Por hacer}: contiene las tareas que no se han empezado.
    \item \textbf{Haciendo}: contiene las tareas en las que se está trabajando actualmente.
    \item \textbf{Hechas}: contiene las tareas que se han completado.
    \item \textbf{Mejoras}: contiene tareas u objetivos que salen del alcance del proyecto actual y que pueden ser desarrolladas para una versión futura del producto.
\end{itemize}

Cada una de las tareas en este tablero puede ser asignada a una persona específica, también se pueden etiquetar dependiendo de la naturaleza de la tarea.

Para el proyecto presente se escribió una lista de objetivos del producto al inicio y se fue actualizando a medida que se avanzó en el desarrollo. 

\section{El equipo de Scrum}
El equipo de Scrum está formado por el Dueño del Producto, el Equipo de Desarrollo y un \emph{Scrum Master}. Los equipos de Scrum son auto-organizados y auto-suficientes. Los equipos auto-organizados escogen la mejor forma de llevar a cabo el trabajo, en vez de ser dirigidos por otras personas fuera del equipo. Los equipos auto-suficientes tienen todas las competencias necesarias para terminar el trabajo sin depender de otros fuera del equipo. \cite{scrumSchwaber} A continuación se describirán los roles de un equipo Scrum:

\subsection{Dueño del Producto}
Es el responsable de maxizimizar el valor del producto desarrollado por el Equipo de Desarrollo. Es el responsable de manejar el \emph{Product Backlog} (pila de trabajo no realizado), esto incluye:

\begin{itemize}
    \item Expresar los items del \emph{Product Backlog}
\end{itemize}


\section{Eventos de Scrum}
